\begin{center}

Plan de Capacitación para Becario ó Becaria 

Plan de Beca A1P 2020

\end{center}

\section*{ Título de la Tesis }

Modelización Atomística de la tenacidad de interfaces Zr / ZrO\textsubscript{2}

\section*{Directores}

\subsubsection{Mariano Daniel Forti}

Doctor en Ciencia y Tecnología, Mención Materiales, Instituto Sabato 
UNSAM-CNEA. Investigador CNEA / Docente UNSAM. 
Contacto: \href{mailto:mforti@cnea.gov.ar}{mforti@cnea.gov.ar}, Interno CAC 7832.

\subsubsection{Pablo Gargano}

Licenciado en Ciencias Físicas. Docente UNSAM e Investigador CNEA.
Contacto: \href{mailto:gargano@cnea.gov.ar}{gargano@cnea.gov.ar}. Interno CAC 7247.


\section*{Descripción sucinta de los conocimientos y habilidades que el/la  
 Becario ó Becara adquirirá durante este trabajo }

\subsection{ Revisión Bibliográfica}

La revisión bibliográfica será un trabajo constante a lo largo de los 13 meses, 
a través de la cual se entrenará en métodos de análisis y búsqueda de 
información.

\subsection{ Cálculos DFT}

Adquirirá experiencia en la aplicación de técnicas de cálculo computacional 
basados en la DFT. Al mismo tiempo aprenderá las relaciones entre los 
parámetros involucrados en los modelos y las cantidades observables 
experimentalmente.

Esta experiencia implica el diseño de la estrategia de cálculo, la 
determinación de los parámetros específicos de acuerdo a su significado teórico 
y el análisis constante de los resultados. La toma de decisiones durante el 
desarrollo de la tesis surgirá tanto del análisis de las mediciones como de la 
comparación crítica de los valores propios con los de la literatura.

\subsection{ Conocimiento del sistema interfacial Zr/ZrO2 }

Habrá ganado conocimiento específico sobre los temas del proyecto, 
microestructura y tenacidad de la interfaz metal / óxido metálico. 

\subsection{ Desarrollo profesional}

Se habrá desempeñado profesionalmente en un grupo de trabajo, con la 
oportunidad de participar de discusiones y de fortalecer su sentido crítico y 
su capacidad colaborativa.

\subsection{ Habilidades de comunicación }

Finalmente, se pondrá énfasis en la habilidad de comunicación de los 
resultados, tanto en la identificación de los datos relevantes como en la 
redacción de los informes. 


\section*{Técnicas experimentales o de cálculo que se usarán}

Los cálculos serán hechos dentro del marco de la DFT \cite{KohnSham65,
HohenbergKohn64} usando el método 
Projector Augmented Wave\cite{Bloch1994,Kresse1999}
implementado en el código Vienna Ab-initio 
Simulation Package\cite{Hafner2007,Hafner2008}
en la Aproximación del Gradiente Generalizado según la 
arametrización PBE \cite{PBE}. 

Se utilizarán herramientas programadas en bash y/o python para procesamiento de 
texto plano y análisis de resultados. Se utilizarán herramientas de conexión 
remota nativas de unix / linux para acceso a las facilidades de cálculo.

\section*{Cronograma estimado de cada una de las actividades}

Se tomará un período de 12 (doce) meses para la capacitación completa
del Becario o Becaria. 

\vspace{1cm}
{
  \rowcolors{4}{gray!20!white}{white!20!gray}
  \renewcommand{\arraystretch}{2}
  
\begin{table}[h!]\footnotesize
  \begin{tabular}{ | p{3cm} | *{12}{m{0.5cm}} | } %c{2cm}|c{2cm}|c{2cm}|c{2cm}|c{2cm}| }
  \hline
    \multicolumn{1}{|c|}{Actividad  } & \multicolumn{12}{c|}{ Mes } \\ [0.5em]
   \cline{2-13}
&1&2&3&4&5&6&7&8&9&10&11&12\\
\hline

\hline
    Revisión bibliográfica & \xmark & \xmark &&&&& \xmark & \xmark & \xmark && \xmark &\\
    Entrenamiento en el código de cálculo& \xmark & \xmark & \xmark &&&&&&&&&\\
    Cálculo de los bloques relajados de metal y óxido por separado&&& \xmark & \xmark & \xmark &&&&&&&\\
    Cálculo de la energía de la interfaz metal/óxido en función de la separación&&&&& \xmark & \xmark & \xmark & \xmark & \xmark & \xmark &&\\
    Análisis de resultados y escritura de Informes &&&& \xmark & \xmark & \xmark & \xmark & \xmark & \xmark & \xmark & \xmark &\xmark \\
%% \end{tabularx}
  \hline
\end{tabular}
\end{table}
}

\section*{ Materiales e Infraestructura con que se cuenta }

El grupo cuenta con licencia para el uso del código de cálculo VASP\cite{Hafner2007,
Hafner2008} , y con 
2 conjuntos de computadoras (clusters), de 32 procesadores cada uno, para 
cálculo de alto rendimiento utilizados en exclusividad por el grupo. Además, se 
cuenta con acceso a las facilidades de cálculo de altas prestaciones de la CNEA, 
que actualmente cuenta con el cluster ISAAC 750 procesadores y el nuevo cluster 
Neurus de 1040 procesadores.

\section*{Experiencia previa del grupo de trabajo en el tema propuesto Aplicaciones 
de DFT}

En la actualidad la División de Aleaciones Especiales trabaja en el marco de un 
PICT 2015 y un Proyecto UNSAM 2017-2019. Estos proyectos implican por un lado 
estudios teóricos de los sistemas Zr metálico y de los defectos puntuales de la 
ZrO2 y la interacción entre ambos sistemas, pero por otro lado se realizan 
estudios experimentales para el diseño de nuevas aleaciones de Zr para uso en 
reactores nucleares de potencia de cuarta generación. 
Por otro lado, los directores poseen amplia experiencia en el estudio de 
sistemas interfaciales mediante métodos computacionales. En sus tesis de 
doctorado se llevaron a cabo estudios de adhesión, defectos puntuales y 
difusión que dieron también lugar a publicaciones en revistas internacionales. 

\subsection{ Tesis desarrolladas en el grupo en temas de cálculos DFT}

\begin{enumerate}


\item Tesis de Doctorado en Ciencia y Tecnología, Mención Materiales, 
Instituto Sabato, UNSAM: “Modelo atomístico/continuo aplicado a la fractura de 
la capa de óxido en tuberías de reactores nucleares de potencia”. Mariano Forti 
, 2017.

\item Tesis de Doctorado UBA, área Ciencias Físicas: “Defectos 
constitucionales y energía de migración de aluminio en UAl4”. Laura Kniznik, 
2016. 

\end{enumerate}

\subsection{ Publicaciones del grupo en temas de cálculo DFT}

\begin{enumerate}
\item  G.E. Ramírez-Caballero, P.B. Balbuena, P. R. Alonso, P.H. Gargano, G. 
H. Rubiolo, ``Carbon Adsorption and Absorption in the (111) L12 Fe3Al Surface'', J. 
Phys. Chem. C 113 (2009) 18321–18330.

\item  P.R. Alonso, J.R. Fernández, P.H. Gargano, G.H. Rubiolo, U-Al system. 
``Ab-initio and many body potential approaches'', Physica B: Condensed Matter 404 
Issue 18 (2009) 2851-2853.

\item P.R. Alonso, P.H. Gargano, G.E. Ramírez-Caballero, P.B. Balbuena, G.H. 
Rubiolo, ``First principles calculation of L21 + A2 coherent equilibria in the 
Fe-Al-Ti system'', Physica B: Condensed Matter 404 Issue 18 (2009) 2845-2847. 

\item  F. Lanzini, P.H. Gargano, P.R. Alonso, G.H. Rubiolo, ``First principles 
study of phase stabilities in bcc Cu-Al alloy'', Modelling Simul. Mater. Sci. 
Eng. 19 (2011) 015008 (15pp).

\item  P.R. Alonso, P.H. Gargano, P.B. Bozzano, G.E. Ramírez-Caballero, P.B. 
Balbuena, G.H. Rubiolo, ``Combined ab initio and experimental study of A2 + L21 
coherent equilibria in the Fe-Al-X (X=Ti, Nb, V) systems'', Intermetallics 19 
Issue 8 (2011) 1157-1167. 

\item  L. Kniznik, P.R. Alonso, P.H. Gargano, G.H. Rubiolo, ``Simulation of 
UAl4 growth in an UAl3/Al diffusion couple'', Journal of Nuclear Materials 414, 
    309-315. 

\item  P.R. Alonso, P.H. Gargano, G.H. Rubiolo, ``Stability of the C14-laves 
phase (Fe,Si)2Mo from ab initio calculations'', Computer Coupling of Phase 
Diagrams and Thermochemistry (Calphad), 35 (2011) 492–498. 

\item  P.R. Alonso, P.H. Gargano, L. Kniznik, L.M. Pizarro, G.H. Rubiolo; 
``Experimental studies and first principles calculations in nuclear fuel alloys 
for research reactors'' in Nuclear Materials; Editor: Michael P. Hemsworth. 
Series: Physics Research and Technology, Materials Science and Technologies. 
Nova Science Publishers, Inc; 400 Oser Avenue, Suite 1600, Hauppauge, NY 11788, 
EEUU. ISBN: 978-1-61324-010-6. (Nova Science Publishers, Inc , New York, 2011).

\item  P.R. Alonso, P.H. Gargano, G.H. Rubiolo, ``First principles calculation 
of the Al3U-Si3U pseudo binary fcc phase equilibrium diagram'', CALPHAD: Computer 
Coupling of Phase Diagrams and Thermochemistry 38 (2012) 117–121.

\item  M.D. Forti, P.R. Alonso, P.H. Gargano, G.H. Rubiolo, ``Transition 
metals monoxides. An LDA+U study'', Procedia Materials Science 1 (2012) 230 – 234.

\item L. Kniznik, P.R. Alonso, P.H. Gargano, M.D. Forti, G.H. Rubiolo, ``First 
principles study of U-Al system ground state'', Procedia Materials Science 1 , 514 – 519.

\item  M.D. Forti, P. Balbuena, P.R. Alonso, ``Ab-initio studies on 
carburization of Fe3Al based alloys'', Procedia Materials Science 1 (2012) 191 – 
198.

\item  M.D. Forti, P.R. Alonso, P.H. Gargano, G.H. Rubiolo, ``Adhesion Energy 
of the Fe(BCC)/Magnetite Interface within the DFT approach'', Procedia Materials 
Science 8 (2015) 1066 – 1072.

\item  P.R. Alonso, ``Aplicaciones de técnicas de primeros principios al 
cálculo de diagramas de fases de equilibrio''. ``Diagramas de fases de equilibrio 
ternarios Fe-Al-V a partir de expansión en cúmulos y método variacional'', 
conferencista invitada en el ``Workshop en Procesamiento Fïsico Químico 
Avanzado'', Bucaramanga, Colombia, Marzo 2014.

\item  P.R. Alonso, ``Aplicaciones de técnicas de primeros principios al 
cálculo de diagramas de fases de equilibrio. Diagrama de fases de equilibrio 
pseudo-binario UAl3\_USi3 a partir de expansión en cúmulos y simulaciones de 
MonteCarlo'', conferencista invitada en el ''Workshop en Procesamiento Fïsico 
Químico Avanzado'', Bucaramanga, Colombia, Marzo 2014.

\item M.D. Forti, dictado del taller ``VASP'', conferencista invitado en el 
“Workshop en Procesamiento Fïsico Químico Avanzado”, Bucaramanga, Colombia, 
Marzo 2014.

\item  M.D. Forti, ``Propiedades Mecánicas'', conferencista invitado en el 
“Workshop en Procesamiento Fïsico Químico Avanzado”, Bucaramanga, Colombia, 
Marzo 2014.

\item  D. Tozini, M. Forti, P.H. Gargano, Cálculo de diferencias de carga en 
interfaces Fe/Fe3O4 a partir de resultados de DFT, Trabajo 5056, 14º Congreso 
Internacional SAM-CONAMET/IBEROMAT XIII /MATERIA, Santa Fé, 2014.

\item  D. Tozini, M.D. Forti, P.H. Gargano, P.R. Alonso, G.H. Rubiolo, 
Charge difference calculation in Fe/Fe3O4 interfaces from DFT results, Procedia 
Materials Science 9 (2015) 612 – 618.

\item L. Kniznik, P.R. Alonso, P.H. Gargano , G.H. Rubiolo, Energetics and 
electronic structure of UAl4 with point defects, Journal of Nuclear Materials 
466 (2015) 539-550.

\item M.D. Forti, P.R. Alonso, P.H. Gargano, P.B. Balbuena, G.H. Rubiolo, A 
DFT study of atomic structure and adhesion at the Fe(BCC)/Fe3O4 interfaces, 
Surface Science 647 (2016) 55–65.

\item  P.H. Gargano, L. Kniznik, P.R. Alonso, M.D. Forti, G.H. Rubiolo, 
Concentration of constitutional and thermal defects in UAl4, Journal of Nuclear 
Materials 478 (2016) 74-82.
\end{enumerate}

\subsection{ Premios recibidos }

\begin{enumerate} 


\item  Mención especial a Mariano Forti en el CONCURSO ESTÍMULO A JÓVENES 
INVESTIGADORES EN CIENCIA Y TECNOLOGÍA DE MATERIALES. Trabajo premiado: Forti 
M., Balbuena P. y Alonso P., ESTUDIOS AB-INITIO SOBRE CARBURACIÓN EN ALEACIONES 
DE BASE Fe3Al, 11º Congreso Binacional de Metalurgia y Materiales SAM / CONAMET 
2011, 18 al 21 de Octubre de 2011 - Rosario, Argentina.

\item  PREMIO JORGE KITTL. Mejor trabajo en Investigación Básica en Ciencia 
de Materiales. Trabajo premiado: Paula R. Alonso, Pablo H.Gargano y Gerardo H. 
Rubiolo, ESTADO FUNDAMENTAL DEL PSEUDO BINARIO Al3U-Si3U POR PRIMEROS 
PRINCIPIOS. SOLUCIÓN SÓLIDA U(Al,Si)3, 11º Congreso Binacional de Metalurgia y 
Materiales SAM / CONAMET 2011, 18 al 21 de Octubre de 2011 - Rosario, Argentina.

\item  PREMIO JORGE KITTL. Mejor trabajo en Investigación Básica en Ciencia 
de Materiales. Trabajo premiado: Pedro A. Ferreirós, Paula R. Alonso, Pablo H. 
Gargano, Patricia B. Bozzano, Horacio E. Troiani y Gerardo H. Rubiolo, 
    Transformaciones de fase en aleaciones $Fe_{1-2X}Al_X V_X$ (X=1,15), XIII Congreso 
Internacional SAM-CONAMET Iguazú 2013, 20 al 23 de agosto de 2013, Iguazú, 
Misiones, Argentina.

\end{enumerate}

\section*{ Proyectos científicos y/o tecnológicos de los que participaría esta tesis  }

\begin{itemize}

\item Los temas a desarrollar en esta Tesis de Maestría son parte de un proyecto
    presentado para su financiación
por ANPCyT en el concurso PICT2018 bajo el título ``Simulación y ensayo de una
aleación propuesta para reactores nucleares de alto
Quemado''.

\end{itemize}

%%%%%  Objetivos Estratégicos

\section{
Objetivos generales y específicos de acuerdo al plan estratégico de CNEA 2015-
2025 en los que se enmarca este proyecto
}

\subsection{Combustibles Nucleares}

\subsubsection{Objetivo Estratégico 1}
Objetivo Estratégico 1: Fortalecer e intensificar la capacidad de investigación,
desarrollo e ingeniería en el campo de los elementos combustibles para
centrales nucleares de potencia y reactores experimentales y de producción
de radioisótopos. 


\subsection{Centrales Nucleares}

\subsubsection{Objetivo Estratégico 2}
Consolidar la autonomía tecnológica de CNEA en
el campo de los reactores nucleares de potencia, posicionándose como la
organización de soporte tecnológico de las centrales nucleares instaladas
en el país y contribuyendo a la sustentabilidad de la operación durante todo
su ciclo de vida.


\subsection{
Las personas, sus conocimientos y su
formación académica
}

\subsubsection{Objetivo Estratégico 1}
Efectuar una gestión estratégica de Recursos
Humanos que contemple la planificación, el desarrollo y la retención del
capital humano.

\subsubsection{Objetivo Estratégico 2}
Asegurar el desarrollo, la preservación y la
transferencia de conocimientos y experiencias, contribuyendo a la
sostenibilidad de la actividad nuclear.

\subsubsection{Objetivo Estratégico 3}
Afianzar las actividades de educación y
capacitación de los Institutos Académicos y otros centros de formación de
CNEA, atendiendo a las necesidades y prioridades para el desarrollo de la
actividad nuclear.

\section{
Áreas temáticas de acuerdo al plan estratégico de CNEA 2015-
2025 en los que se enmarca este proyecto
}

\subsection{Combustibles Nucleares}

\subsubsection{Objetivo Estratégico 1}

Mantener y acrecentar la autonomía tecnológica
para el desarrollo, el diseño, la ingeniería y la fabricación de los elementos
combustibles para las centrales nucleares argentinas, actuales y futuras.

\paragraph{Objetivo Específico 1.1}
Desarrollar los elementos combustibles para las centrales CAREM.
%
\paragraph{Objetivo Específico 1.2} Optimizar el diseño de los elementos combustibles para las
centrales nucleares PHWR argentinas.

\paragraph{Objetivo Específico 1.4}

Avanzar en el desarrollo de la ingeniería y la tecnología de
fabricación de elementos combustibles para las centrales nucleares PWR.


\subsection{Reactores de Potencia}

\subsubsection{Objetivo Estratégico 2}

Ser la Organización de Soporte Tecnológico (TSO)
de las centrales nucleares, proveyendo asistencia tecnológica en diseño,
licenciamiento, construcción, operación y desmantelamiento.

\paragraph{Objetivo Específico 2.1}
Mantener e incrementar la capacidad en investigación y
desarrollo tecnológico y en áreas de ingeniería especializadas.

\paragraph{Objetivo Específico 2.2}

Implementar capacidades de ingeniería para evaluar las nuevas
tecnologías y propender a una participación relevante en los proyectos de las futuras
centrales.

\subsubsection{Objetivo Estratégico 2}

Ser la Organización de Soporte Tecnológico (TSO)
de las centrales nucleares, proveyendo asistencia tecnológica en diseño,
licenciamiento, construcción, operación y desmantelamiento.

\paragraph{Objetivo Específico 2.3}

Actualizar en forma permanente la información tecnológica de
las centrales nucleares en todas sus etapas y optimizar su uso.

\subsection{investigación y desarrollo}
\subsubsection{investigación y desarrollo en física}
\paragraph{objetivo estratégico 1}
generar conocimientos y tecnologías vinculadas con las ciencias físicas. (pg 132)

\begin{itemize}

\item objetivo específico 1.1

  Mantener e incrementar la capacidad en investigación y desarrollo en
  materiales de uso nuclear: nuevos combustibles, materiales estructurales y
  funcionales, barreras de contención y matrices para inmovilización de
  residuos radioactivos.

\item objetivo específico 1.3

  Generar conocimiento en materia condensada y física de superficies.

\end{itemize}

\subsubsection{investigación y desarrollo en materiales y ensayos no destructivos}

\paragraph{objetivo estratégico 1}

Generar y aplicar conocimiento científico y tecnológico, en el área de
materiales y ensayos no destructivos y estructurales, para atender los
requerimientos de cnea y posibilitar su transferencia a otros sectores
tecnológicos.

\begin{itemize}

\item  objetivo específico 1.6

  Ampliar el conocimiento y desarrollar
nuevos materiales y recubrimientos de interés para el sector tecnológico

\end{itemize}


\bibliography{Bibliography}
\bibliographystyle{ieeetr}

%\vspace{2cm}
%\begin{table}[h!]
%  \begin{tabular*}{\textwidth}{ *{2}{>\centering p{0.5\textwidth} } }
%    &\\
%    \hline
%    Mariano Forti & Pablo Gargano \\
%  \end{tabular*}
%\end{table}

