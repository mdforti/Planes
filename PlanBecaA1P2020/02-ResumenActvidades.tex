 % lo saco de la sección " trabajos a realizar " del plan de Maestría. 

\subsection{ Revisión Bibliográfica. }

De esta revisión deben decidirse la relación de orientaciones más probable 
entre los constituyentes de la interfaz. Por otro lado debe hacerse un recuento 
de las áreas de incidencia tecnológica del problema. También se deberá hacer 
una búsqueda constante durante todo el periodo sobre nuevos datos 
experimentales de adhesión en el sistema de interés

\subsection{ Construcción del modelo atomístico de la interfaz.}

A partir de las relaciones de orientaciones obtenida de la revisión 
bibliográfica, se construirán los modelos de interfaz para las posibles 
configuraciones de apilamiento a nivel atomístico, entre las posibilidades que 
no se puedan discernir experimentalmente. Se construirán los bloques 
superficiales en función de estos datos haciendo un estudio de convergencia de 
las propiedades superficiales en función del número de capas atómicas y los 
parámetros de cálculo. Con los bloques superficiales se armará por apilamiento 
los modelos de interfaz. Utilizando los resultados de los cálculos de energía 
en función de la separación interfacial para las distintas configuraciones 
\cite{Jiangetal} 
se decidirá la interfaz más adherida.

\section{Técnicas experimentales o de cálculo que se usarán}

Los cálculos serán hechos dentro del marco de la DFT \cite{KohnSham65,
HohenbergKohn64} usando el método 
Projector Augmented Wave\cite{Bloch1994,Kresse1999}
implementado en el código Vienna Ab-initio 
Simulation Package\cite{Hafner2007,Hafner2008}
en la Aproximación del Gradiente Generalizado según la 
arametrización PBE \cite{PBE}. 

Se utilizarán herramientas programadas en bash y/o python para procesamiento de 
texto plano y análisis de resultados. Se utilizarán herramientas de conexión 
remota nativas de unix / linux para acceso a las facilidades de cálculo.

