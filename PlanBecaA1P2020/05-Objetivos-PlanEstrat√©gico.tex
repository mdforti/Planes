\section{
Objetivos generales y específicos de acuerdo al plan estratégico de CNEA 2015-
2025 en los que se enmarca este proyecto
}

\subsection{Combustibles Nucleares}

\subsubsection{Objetivo Estratégico 1}
Objetivo Estratégico 1: Fortalecer e intensificar la capacidad de investigación,
desarrollo e ingeniería en el campo de los elementos combustibles para
centrales nucleares de potencia y reactores experimentales y de producción
de radioisótopos. 


\subsection{Centrales Nucleares}

\subsubsection{Objetivo Estratégico 2}
Consolidar la autonomía tecnológica de CNEA en
el campo de los reactores nucleares de potencia, posicionándose como la
organización de soporte tecnológico de las centrales nucleares instaladas
en el país y contribuyendo a la sustentabilidad de la operación durante todo
su ciclo de vida.


\subsection{
Las personas, sus conocimientos y su
formación académica
}

\subsubsection{Objetivo Estratégico 1}
Efectuar una gestión estratégica de Recursos
Humanos que contemple la planificación, el desarrollo y la retención del
capital humano.

\subsubsection{Objetivo Estratégico 2}
Asegurar el desarrollo, la preservación y la
transferencia de conocimientos y experiencias, contribuyendo a la
sostenibilidad de la actividad nuclear.

\subsubsection{Objetivo Estratégico 3}
Afianzar las actividades de educación y
capacitación de los Institutos Académicos y otros centros de formación de
CNEA, atendiendo a las necesidades y prioridades para el desarrollo de la
actividad nuclear.

\section{
Áreas temáticas de acuerdo al plan estratégico de CNEA 2015-
2025 en los que se enmarca este proyecto
}

\subsection{Combustibles Nucleares}

\subsubsection{Objetivo Estratégico 1}

Mantener y acrecentar la autonomía tecnológica
para el desarrollo, el diseño, la ingeniería y la fabricación de los elementos
combustibles para las centrales nucleares argentinas, actuales y futuras.

\paragraph{Objetivo Específico 1.1}
Desarrollar los elementos combustibles para las centrales CAREM.
%
\paragraph{Objetivo Específico 1.2} Optimizar el diseño de los elementos combustibles para las
centrales nucleares PHWR argentinas.

\paragraph{Objetivo Específico 1.4}

Avanzar en el desarrollo de la ingeniería y la tecnología de
fabricación de elementos combustibles para las centrales nucleares PWR.


\subsection{Reactores de Potencia}

\subsubsection{Objetivo Estratégico 2}

Ser la Organización de Soporte Tecnológico (TSO)
de las centrales nucleares, proveyendo asistencia tecnológica en diseño,
licenciamiento, construcción, operación y desmantelamiento.

\paragraph{Objetivo Específico 2.1}
Mantener e incrementar la capacidad en investigación y
desarrollo tecnológico y en áreas de ingeniería especializadas.

\paragraph{Objetivo Específico 2.2}

Implementar capacidades de ingeniería para evaluar las nuevas
tecnologías y propender a una participación relevante en los proyectos de las futuras
centrales.

\subsubsection{Objetivo Estratégico 2}

Ser la Organización de Soporte Tecnológico (TSO)
de las centrales nucleares, proveyendo asistencia tecnológica en diseño,
licenciamiento, construcción, operación y desmantelamiento.

\paragraph{Objetivo Específico 2.3}

Actualizar en forma permanente la información tecnológica de
las centrales nucleares en todas sus etapas y optimizar su uso.

# AREAS TEMÁTICAS
	- INVESTIGACIÓN Y DESARROLLO (pg 132)
		- INVESTIGACIÓN Y DESARROLLO EN FÍSICA (pg 132)
			- Objetivo Estratégico 1: Generar conocimientos y tecnologías vinculadas con las ciencias físicas. (pg 132)
				- Objetivo Específico 1.3: Generar conocimiento en materia condensada y física de superficies (pg 133)
		- INVESTIGACIÓN Y DESARROLLO EN MATERIALES Y ENSAYOS NO DESTRUCTIVOS (pg 137)
			- Objetivo Estratégico 1: Generar y aplicar conocimiento científico y tecnológico, en el área de materiales y ensayos no destructivos y estructurales, para atender los requerimientos de CNEA y posibilitar su transferencia a otros sectores tecnológicos.
				- Objetivo Específico 1.1: Mantener e incrementar la capacidad en investigación y desarrollo en materiales de uso nuclear: nuevos combustibles, materiales estructurales y funcionales, barreras de contención y matrices para inmovilización de residuos radioactivos(pg 137)
				- Objetivo Específico 1.6: Ampliar el conocimiento y desarrollar nuevos materiales y recubrimientos de interés para el sector tecnológico (pg 138)
	- INSTITUTOS ACADÉMICOS (pg 168)
		- Objetivo General 1: Afianzar las actividades de educación y capacitación de los Institutos Académicos de CNEA, atendiendo a las necesidades e intereses del sistema nuclear argentino (pg 168)
			- Objetivo Particular 1.2: Fomentar actividades de educación permanente y la vinculación con los diferentes niveles de educación formal



