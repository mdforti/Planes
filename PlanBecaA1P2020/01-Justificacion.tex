\subsection{Proyecto institucional Marco}
\begin{itemize}
  \item { Área Temática Combustibles Nucleares, Objetivo Estratégico 1 }
    Mantener y acrecentar la autonomía tecnológica para el desarrollo, el diseño, la ingeniería y la fabricación de los elementos combustibles para las centrales nucleares argentinas, actuales y futuras

  \item {Área temática Centrales Nucleares, Objervio Estratégico 2}
onsolidar la autonomía tecnológica de CNEA en el campo de los reactores nucleares de potencia, posicionándose como la organización de soporte tecnológico de las centrales nucleares instaladas en el país y contribuyendo a la sustentabilidad de la operación durante todo su ciclo de vida. 

\item {Investigación y desarrollo en Física, Objetivo Específico 1.3}
Generar conocimiento en materia condensada y física de superficies.

\end{itemize}


\subsection{Justificación}

El uso de los reactores nucleares como fuente de energía se ha extendido y 
consolidado en la segunda mitad del siglo XX y en los inicios de este siglo 
XXI. La continuidad de este recurso depende en gran medida de que las plantas 
nucleares sean cada vez más seguras y confiables, lo cual implica un 
mejoramiento de todos los aspectos relacionados con su operación. Entre ellos 
se cuenta la integridad de los elementos estructurales, que depende fuertemente 
de la estabilidad mecánica de los óxidos protectores. En las centrales CNA-I y 
CNA-II el Zircaloy-4 (aleación base Zr con adiciones de Sn, Fe and Cr) se 
utiliza en los canales refrigerantes. Sin embargo, esta 
aleación presenta 
problemas de corrosión que, por ejemplo, motivaron el reemplazo de los canales 
refrigerantes de CNA-I en 1988.

La capa de óxido que se forma sobre la superficie metálica se comporta como una 
protección contra la absorción de hidrógeno y la degradación. Si la capa de 
óxido se rompe por la formación de fisuras, la película de óxido cercana a la 
superficie se enriquece en hidrógeno favoreciendo la difusión hacia el metal en 
detrimento de su resistencia mecánica. La formación de fisuras puede ocurrir 
debido a la transformación del óxido tetragonal a monoclínico dado que dicha 
transformación involucra cambios de volumen que pueden inducir tensiones 
internas. El objetivo de minimizar los efectos de dicha transformación requiere 
el control de la textura y de la composición de la superficie metálica. Una 
estrategia para evaluar la resistencia de la interfaz metal/óxido, consiste en 
el cálculo del trabajo de separación de dicha interfaz.

El desprendimiento es función del espesor de la capa de óxido\cite{Schutze2005},
pero el modo 
de falla de la capa de óxido depende del espesor de la misma y del estado de 
tensiones del sistema metal-óxido por lo que el conocimiento de las propiedades 
de la interfaz es de vital importancia. Debe conocerse la tenacidad de la 
interfaz Zr/ZrO2 (tetragonal) y el tamaño crítico del defecto desde donde 
comenzará la fisura. Existen pocos datos fiables experimentales de la tenacidad 
de la interfaz metal/óxido. La alternativa que utilizamos se basa en un cálculo 
teórico a escala atomística, la Teoría de la Funcional Densidad (DFT) 
\cite{KohnSham65,HohenbergKohn64}. La 
investigación teórica de esta interfaz podría dar condiciones de preferencia de 
diseño de la superficie metálica para disminuir la probabilidad de falla.

