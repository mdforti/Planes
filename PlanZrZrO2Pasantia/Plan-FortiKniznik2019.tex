\begin{center}

Plan de Pasantía 2020

\end{center}

\section{ Título de la Pasantía }

Modelización atómística de la energía de fractura de ZrO\textsubscript{2 }

\section{ Lugar en donde se realizará }
División Aleaciones Especiales, Gerencia Materiales, Gerencia de Área Energía 
Nuclear

\section{Directores}

\subsubsection{Mariano Daniel Forti}

Doctor en Ciencia y Tecnología, Mención Materiales, Instituto Sabato 
UNSAM-CNEA. Investigador CNEA / Docente UNSAM. 
Contacto: \href{mailto:mforti@cnea.gov.ar}{mforti@cnea.gov.ar}, Interno CAC 7832.

\subsubsection{Laura Kniznik}

Doctora en Ciencias Físicas, Universidad de Buenos Aires. Docente UNSAM e Investigadora CNEA.
Contacto: \href{mailto:kniznik@cnea.gov.ar}{kniznik@cnea.gov.ar}. Interno CAC 7832.

\section{Perfil Preferido del estudiante}

Se busca una persona con buena predisposición para el trabajo
en equipo e iniciativa para desarrollar estrategias de 
trabajo. Será de utilidad que pueda leer e interpretar
material en inglés.  

\section{Introducción }

El uso de los reactores nucleares como fuente de energía se ha extendido y 
consolidado en la segunda mitad del siglo XX y en los inicios de este siglo 
XXI. La continuidad de este recurso depende en gran medida de que las plantas 
nucleares sean cada vez más seguras y confiables, lo cual implica un 
mejoramiento de todos los aspectos relacionados con su operación. Entre ellos 
se cuenta la integridad de los elementos estructurales, que depende fuertemente 
de la estabilidad mecánica de los óxidos protectores. En las centrales CNA-I y 
CNA-II el Zircaloy-6 (aleación base Zr con adiciones de Sn, Fe and Cr) se 
utiliza en los canales refrigerantes\cite{Estevez2000}. Sin embargo, esta 
aleación presenta 
problemas de corrosión que, por ejemplo, motivaron el reemplazo de los canales 
refrigerantes de CNA-I en 1988.

La capa de óxido que se forma sobre la superficie metálica se comporta como una 
protección contra la absorción de hidrógeno y la degradación. Si la capa de 
óxido se rompe por la formación de fisuras, la película de óxido cercana a la 
superficie se enriquece en hidrógeno favoreciendo la difusión hacia el metal en 
detrimento de su resistencia mecánica. La formación de fisuras puede ocurrir 
debido a la transformación del óxido tetragonal a monoclínico dado que dicha 
transformación involucra cambios de volumen que pueden inducir tensiones 
internas. El objetivo de minimizar los efectos de dicha transformación requiere 
el control de la textura y de la composición de la superficie metálica.
Una estrategia para evaluar la resistencia de la capa de óxido consiste en 
conocer los mecanismos de fractura de la zirconia tetragonal y monoclínica. 

El desprendimiento del óxido es función de su espesor \cite{Schutze2005},
y de su estado de tensiones. Deben conocerse las propiedades
mecánicas de las fases de ZrO\textsubscript{2} presentes en la 
capa de óxido. Si bien los óxodos pueden ser estudiados 
experimentalmente , la Teoría de la Funcional Densidad (DFT) 
\cite{KohnSham65,HohenbergKohn64} posibilita correlacionar las
propiedades mecánicas del cristal con los fenómenos electronicos 
y los enlaces químicos que le dan sus propiedades. Por otro lado, conocer
las propiedades superficiales es el escalón inicial hacia el conocimiento
de las propiedades de la interfaz metal / óxido y óxido/medio. 

\section{Objetivos del Trabajo}

Utilizando la Teoría de la Funcional Densidad, calcular la energía de fractura de la
ZrO\textsubscript{2} monoclínica y tetragonal, en una variedad de planos cristalinos de bajo índice.
Realizar un análisis termodinámico a partir de los resultados de los 
cálculos de primeros principios  para decidir sobre las superficies más
estables de cada cristal. 

\section{Descripción sucinta de los conocimientos y habilidades que el/la  
pasante adquirirá durante este trabajo }

\subsection{ Revisión Bibliográfica}

La revisión bibliográfica será un trabajo constante a lo largo de los 13 meses, 
a través de la cual se entrenará en métodos de análisis y búsqueda de 
información.

\subsection{ Cálculos DFT}

Adquirirá experiencia en la aplicación de técnicas de cálculo computacional 
basados en la DFT. Al mismo tiempo aprenderá las relaciones entre los 
parámetros involucrados en los modelos y las cantidades observables 
experimentalmente.

Esta experiencia implica el diseño de la estrategia de cálculo, la 
determinación de los parámetros específicos de acuerdo a su significado teórico 
y el análisis constante de los resultados. La toma de decisiones durante el 
desarrollo de la pasantía surgirá tanto del análisis de los cálculos
como de la 
comparación crítica de los valores propios con los de la literatura.

\subsection{ Conocimiento de las distintas fases de ZrO\textsubscript{2} en 
componentes estructurales de plantas nucleares de potencia}

Habrá ganado conocimiento específico sobre los temas del proyecto, 
microestructura y propiedades de la capa de óxido en los combustibles nucleares. 

\subsection{ Desarrollo profesional}

Se habrá desempeñado profesionalmente en un grupo de trabajo, con la 
oportunidad de participar de discusiones fortaleciendo su sentido crítico y su capacidad colaborativa.

\subsection{ Habilidades de comunicación }

Finalmente, se pondrá énfasis en la habilidad de comunicación de los 
resultados, tanto en la identificación de los datos relevantes como en la 
redacción de los informes. 

\section{Trabajos a realizar}

\subsection{ Revisión Bibliográfica. }

Debe hacerse un recuento de las áreas de incidencia tecnológica del 
problema. 
De esta revisión deben decidirse las orientaciones cristalinas de las 
superficies a calcular según la relevancia . 
También se deberá hacer 
una búsqueda constante durante todo el periodo sobre nuevos datos 
experimentales de adhesión en el sistema de interés

\subsection{ Construcción del modelo atomístico del material en volumen y su superficie.}

A partir de la información en la literatura y bases de datos 
se prepararán los modelos a escala atomística del ZrO\textsubscript{2} 
tetragonal y monocínico. Se calcularán las propiedades básicas para 
reproducir resultados anteriores del grupo. 

Usando los datos de los materiales en volumen, 
se calcularán las propiedades superficiales en función del número de 
capas atómicas y los parámetros de cálculo para diversas orientaciones 
cristalinas de la superficie. Una vez ajustados estos parámetros se 
calculará la energía de fractura para cada caso  
\cite{Jiangetal}. Se evaluará la posibilidad de decidir sobre la
estabilidad relativa de las superficies
calculadas. 

\section{Técnicas experimentales o de cálculo que se usarán}

Los cálculos serán hechos dentro del marco de la DFT \cite{KohnSham65,
HohenbergKohn64} usando el método 
Projector Augmented Wave\cite{Bloch1994,Kresse1999}
implementado en el código Vienna Ab-initio 
Simulation Package\cite{Hafner2007,Hafner2008}
en la Aproximación del Gradiente Generalizado según la 
parametrización PBE \cite{PBE}. 

Se utilizarán herramientas programadas en bash y/o python para procesamiento de 
texto plano y análisis de resultados. Se utilizarán herramientas de conexión 
remota nativas de unix / linux para acceso a las facilidades de cálculo.

\section{Cronograma estimado de cada una de las actividades}

El trabajo podrá completarse en 5 meses.

\vspace{1cm}
{
  \rowcolors{4}{gray!20!white}{white!20!gray}
  \renewcommand{\arraystretch}{2}
\begin{table}[H]\footnotesize
 \centering 
  \begin{tabular}{ | p{8cm} *{5}{|m{0.5cm} } | } %c{2cm}|c{2cm}|c{2cm}|c{2cm}|c{2cm}| }
  \hline
    \multicolumn{1}{|c|}{Actividad  } & \multicolumn{5}{c|}{ Mes } \\ [0.5em]
   \cline{2-6}
&1&2&3&4&5\\
\hline

\hline
   Revisión bibliográfica &\xmark&\xmark&&&\\
   Entrenamiento en el código de cálculo&\xmark&\xmark&&&\\
   Elección de los planos cristalinos para generar la fractura&\xmark&\xmark&&&\\
   Cálculo de los bloques relajados del óxido&&\xmark&\xmark&&\\
   Cálculo de energías de fractura&&&\xmark&\xmark&\xmark\\ 
%% \end{tabularx}
  \hline
\end{tabular}
\end{table}
}

\section{ Materiales e Infraestructura con que se cuenta }

El grupo cuenta con licencia para el uso del código de cálculo VASP\cite{Hafner2007,
Hafner2008} , y con 
2 conjuntos de computadoras (clusters), de 32 procesadores cada uno, para 
cálculo de alto rendimiento utilizados en exclusividad por el grupo. Además, se 
cuenta con acceso a las facilidades de cálculo de altas prestaciones de la CNEA, 
que actualmente cuenta con el cluster ISAAC de 750 procesadores y el 
nuevo cluster Neurus de 1040 procesadores.

\section{Experiencia previa del grupo de trabajo en el tema 
propuesto Aplicaciones de DFT}

En la actualidad la División de Aleaciones Especiales trabaja en el marco 
de un 
PICT 2015 y un Proyecto UNSAM 2017-2019. Estos proyectos implican por un 
lado 
estudios teóricos de los sistemas Zr metálico y de los defectos puntuales 
del 
ZrO\textsubscript{2} y la interacción entre ambos sistemas, pero por otro lado se realizan 
estudios experimentales para el diseño de nuevas aleaciones de Zr para 
uso en 
reactores nucleares de potencia de cuarta generación. 
Por otro lado, los directores poseen amplia experiencia en el estudio de 
sistemas interfaciales mediante métodos computacionales. En sus tesis de 
doctorado se llevaron a cabo estudios de adhesión, defectos puntuales y 
difusión que dieron también lugar a publicaciones en revistas internacionales. 

\subsection{ Tesis desarrolladas en el grupo en temas de cálculos DFT}

\begin{enumerate}


\item Tesis de Doctorado en Ciencia y Tecnología, Mención Materiales, 
Instituto Sabato, UNSAM: “Modelo atomístico/continuo aplicado a la fractura de 
la capa de óxido en tuberías de reactores nucleares de potencia”. Mariano Forti 
, 2017.

\item Tesis de Doctorado UBA, área Ciencias Físicas: “Defectos 
constitucionales y energía de migración de aluminio en UAl4”. Laura Kniznik, 
2016. 

\end{enumerate}

\subsection{ Publicaciones del grupo en temas de cálculo DFT}

\begin{enumerate}
\item  G.E. Ramírez-Caballero, P.B. Balbuena, P. R. Alonso, P.H. Gargano, G. 
H. Rubiolo, ``Carbon Adsorption and Absorption in the (111) L12 Fe3Al Surface'', J. 
Phys. Chem. C 113 (2009) 18321–18330.

\item  P.R. Alonso, J.R. Fernández, P.H. Gargano, G.H. Rubiolo, U-Al system. 
``Ab-initio and many body potential approaches'', Physica B: Condensed Matter 404 
Issue 18 (2009) 2851-2853.

\item P.R. Alonso, P.H. Gargano, G.E. Ramírez-Caballero, P.B. Balbuena, G.H. 
Rubiolo, ``First principles calculation of L21 + A2 coherent equilibria in the 
Fe-Al-Ti system'', Physica B: Condensed Matter 404 Issue 18 (2009) 2845-2847. 

\item  F. Lanzini, P.H. Gargano, P.R. Alonso, G.H. Rubiolo, ``First principles 
study of phase stabilities in bcc Cu-Al alloy'', Modelling Simul. Mater. Sci. 
Eng. 19 (2011) 015008 (15pp).

\item  P.R. Alonso, P.H. Gargano, P.B. Bozzano, G.E. Ramírez-Caballero, P.B. 
Balbuena, G.H. Rubiolo, ``Combined ab initio and experimental study of A2 + L21 
coherent equilibria in the Fe-Al-X (X=Ti, Nb, V) systems'', Intermetallics 19 
Issue 8 (2011) 1157-1167. 

\item  L. Kniznik, P.R. Alonso, P.H. Gargano, G.H. Rubiolo, ``Simulation of 
UAl4 growth in an UAl3/Al diffusion couple'', Journal of Nuclear Materials 414, 
    309-315. 

\item  P.R. Alonso, P.H. Gargano, G.H. Rubiolo, ``Stability of the C14-laves 
phase (Fe,Si)2Mo from ab initio calculations'', Computer Coupling of Phase 
Diagrams and Thermochemistry (Calphad), 35 (2011) 492–498. 

\item  P.R. Alonso, P.H. Gargano, L. Kniznik, L.M. Pizarro, G.H. Rubiolo; 
``Experimental studies and first principles calculations in nuclear fuel alloys 
for research reactors'' in Nuclear Materials; Editor: Michael P. Hemsworth. 
Series: Physics Research and Technology, Materials Science and Technologies. 
Nova Science Publishers, Inc; 400 Oser Avenue, Suite 1600, Hauppauge, NY 11788, 
EEUU. ISBN: 978-1-61324-010-6. (Nova Science Publishers, Inc , New York, 2011).

\item  P.R. Alonso, P.H. Gargano, G.H. Rubiolo, ``First principles calculation 
of the Al3U-Si3U pseudo binary fcc phase equilibrium diagram'', CALPHAD: Computer 
Coupling of Phase Diagrams and Thermochemistry 38 (2012) 117–121.

\item  M.D. Forti, P.R. Alonso, P.H. Gargano, G.H. Rubiolo, ``Transition 
metals monoxides. An LDA+U study'', Procedia Materials Science 1 (2012) 230 – 234.

\item L. Kniznik, P.R. Alonso, P.H. Gargano, M.D. Forti, G.H. Rubiolo, ``First 
principles study of U-Al system ground state'', Procedia Materials Science 1 , 514 – 519.

\item  M.D. Forti, P. Balbuena, P.R. Alonso, ``Ab-initio studies on 
carburization of Fe3Al based alloys'', Procedia Materials Science 1 (2012) 191 – 
198.

\item  M.D. Forti, P.R. Alonso, P.H. Gargano, G.H. Rubiolo, ``Adhesion Energy 
of the Fe(BCC)/Magnetite Interface within the DFT approach'', Procedia Materials 
Science 8 (2015) 1066 – 1072.

\item  P.R. Alonso, ``Aplicaciones de técnicas de primeros principios al 
cálculo de diagramas de fases de equilibrio''. ``Diagramas de fases de equilibrio 
ternarios Fe-Al-V a partir de expansión en cúmulos y método variacional'', 
conferencista invitada en el ``Workshop en Procesamiento Fïsico Químico 
Avanzado'', Bucaramanga, Colombia, Marzo 2014.

\item  P.R. Alonso, ``Aplicaciones de técnicas de primeros principios al 
cálculo de diagramas de fases de equilibrio. Diagrama de fases de equilibrio 
pseudo-binario UAl3\_USi3 a partir de expansión en cúmulos y simulaciones de 
MonteCarlo'', conferencista invitada en el ''Workshop en Procesamiento Fïsico 
Químico Avanzado'', Bucaramanga, Colombia, Marzo 2014.

\item M.D. Forti, dictado del taller ``VASP'', conferencista invitado en el 
“Workshop en Procesamiento Fïsico Químico Avanzado”, Bucaramanga, Colombia, 
Marzo 2014.

\item  M.D. Forti, ``Propiedades Mecánicas'', conferencista invitado en el 
“Workshop en Procesamiento Fïsico Químico Avanzado”, Bucaramanga, Colombia, 
Marzo 2014.

\item  D. Tozini, M. Forti, P.H. Gargano, Cálculo de diferencias de carga en 
interfaces Fe/Fe3O4 a partir de resultados de DFT, Trabajo 5056, 14º Congreso 
Internacional SAM-CONAMET/IBEROMAT XIII /MATERIA, Santa Fé, 2014.

\item  D. Tozini, M.D. Forti, P.H. Gargano, P.R. Alonso, G.H. Rubiolo, 
Charge difference calculation in Fe/Fe3O4 interfaces from DFT results, Procedia 
Materials Science 9 (2015) 612 – 618.

\item L. Kniznik, P.R. Alonso, P.H. Gargano , G.H. Rubiolo, Energetics and 
electronic structure of UAl4 with point defects, Journal of Nuclear Materials 
466 (2015) 539-550.

\item M.D. Forti, P.R. Alonso, P.H. Gargano, P.B. Balbuena, G.H. Rubiolo, A 
DFT study of atomic structure and adhesion at the Fe(BCC)/Fe3O4 interfaces, 
Surface Science 647 (2016) 55–65.

\item  P.H. Gargano, L. Kniznik, P.R. Alonso, M.D. Forti, G.H. Rubiolo, 
Concentration of constitutional and thermal defects in UAl4, Journal of Nuclear 
Materials 478 (2016) 74-82.
\end{enumerate}

\subsection{ Premios recibidos }

\begin{enumerate} 


\item  Mención especial a Mariano Forti en el CONCURSO ESTÍMULO A JÓVENES 
INVESTIGADORES EN CIENCIA Y TECNOLOGÍA DE MATERIALES. Trabajo premiado: Forti 
M., Balbuena P. y Alonso P., ESTUDIOS AB-INITIO SOBRE CARBURACIÓN EN ALEACIONES 
DE BASE Fe3Al, 11º Congreso Binacional de Metalurgia y Materiales SAM / CONAMET 
2011, 18 al 21 de Octubre de 2011 - Rosario, Argentina.

\item  PREMIO JORGE KITTL. Mejor trabajo en Investigación Básica en Ciencia 
de Materiales. Trabajo premiado: Paula R. Alonso, Pablo H.Gargano y Gerardo H. 
Rubiolo, ESTADO FUNDAMENTAL DEL PSEUDO BINARIO Al3U-Si3U POR PRIMEROS 
PRINCIPIOS. SOLUCIÓN SÓLIDA U(Al,Si)3, 11º Congreso Binacional de Metalurgia y 
Materiales SAM / CONAMET 2011, 18 al 21 de Octubre de 2011 - Rosario, Argentina.

\item  PREMIO JORGE KITTL. Mejor trabajo en Investigación Básica en Ciencia 
de Materiales. Trabajo premiado: Pedro A. Ferreirós, Paula R. Alonso, Pablo H. 
Gargano, Patricia B. Bozzano, Horacio E. Troiani y Gerardo H. Rubiolo, 
    Transformaciones de fase en aleaciones $Fe_{1-2X}Al_X V_X$ (X=1,15), XIII Congreso 
Internacional SAM-CONAMET Iguazú 2013, 20 al 23 de agosto de 2013, Iguazú, 
Misiones, Argentina.

\end{enumerate}

\section{ Proyectos científicos y/o tecnológicos de los que participaría esta pasantía  }

Los temas a desarrollar en esta pasantía son parte de un proyecto
    presentado para su financiación
por ANPCyT en el concurso PICT2018 bajo el título ``Simulación y ensayo de una
aleación propuesta para reactores nucleares de alto
Quemado''.




\bibliography{Bibliography}
\bibliographystyle{ieeetr}

\vspace{2cm}
\begin{table}[h!]
  \begin{tabular*}{\textwidth}{ *{2}{>\centering p{0.5\textwidth} } }
    \hline
    Mariano Forti & Laura Kniznik  \\
  \end{tabular*}
\end{table}

