\begin{center}

Plan para Trabajo Final de Ingeniería 2021

\end{center}

\section{ Título del Trabajo }

Modelización Atomística de los procesos de fractura del ZrO\textsubscript{2}

\section{ Lugar en donde se realizará }
División Aleaciones Especiales, Gerencia Materiales, Gerencia de Área Energía 
Nuclear

\section{Directores}

\subsubsection{Mariano Daniel Forti}

Doctor en Ciencia y Tecnología, Mención Materiales, Instituto Sabato 
UNSAM-CNEA. Investigador CNEA / Docente UNSAM. 
Contacto: \href{mailto:mforti@cnea.gov.ar}{mforti@cnea.gov.ar}, Interno CAC 7832.

\subsubsection{Pablo Gargano}

Licenciado en Ciencias Físicas, UBA.Docente UNSAM e Investigador CNEA.
Contacto: \href{mailto:gargano@cnea.gov.ar}{gargano@cnea.gov.ar}.

\section{Perfil Preferido del Estudiante para este estudio}

Estudiante el último semestre de Ingeniería en materiales.
Se busca una persona con buena predisposición para el trabajo
en equipo e iniciativa para desarrollar estrategias de 
trabajo. Será de utilidad que pueda leer e interpretar
material en inglés.  

\section{Introducción }

El uso de los reactores nucleares como fuente de energía se ha extendido y 
consolidado en la segunda mitad del siglo XX y en los inicios de este siglo 
XXI. La continuidad de este recurso depende en gran medida de que las plantas 
nucleares sean cada vez más seguras y confiables, lo cual implica un 
mejoramiento de todos los aspectos relacionados con su operación. Entre ellos 
se cuenta la integridad de los elementos estructurales, que depende fuertemente 
de la estabilidad mecánica de los óxidos protectores. En las centrales CNA-I y 
CNA-II el Zircaloy-4 (aleación base Zr con adiciones de Sn, Fe and Cr) se 
utiliza en los canales refrigerantes\cite{Estevez2000}. Sin embargo, esta 
aleación presenta 
problemas de corrosión que, por ejemplo, motivaron el reemplazo de los canales 
refrigerantes de CNA-I en 1988.

La capa de óxido que se forma sobre la superficie metálica se comporta como una 
protección contra la absorción de hidrógeno y la degradación. Si la capa de 
óxido se rompe por la formación de fisuras, la película de óxido cercana a la 
superficie se enriquece en hidrógeno favoreciendo la difusión hacia el metal en 
detrimento de su resistencia mecánica\cite{Schutze2005}.
La formación de fisuras puede ocurrir 
debido a la transformación del óxido tetragonal (t-ZrO\textsubscript{2} ) 
a monoclínico (m-ZrO\textsubscript{2}) \cite{Li2007, Motta2011}  dado que dicha 
transformación involucra cambios de volumen que pueden inducir tensiones 
internas. El objetivo de minimizar los efectos de dicha transformación requiere 
el control de la textura y de la composición de la superficie metálica. Una 
estrategia para evaluar la resistencia del óxido consiste en 
el la estimación de las propiedades mecáncias de las fases presentes
y de los mecanismos de fractura intervinientes.

El desprendimiento del óxido es función del espesor de la capa de óxido\cite{Schutze2005},
pero el modo de falla de la capa protectora depende del espesor de la misma y del estado de 
tensiones del sistema metal-óxido por lo que el conocimiento de las propiedades 
de la interfaz es de vital importancia. Debe conocerse la tenacidad de la 
interfaz Zr/ZrO\textsubscript{2}  pero también las interfaz
m-ZrO\textsubscript{2}/t-ZrO\textsubscript{2}y el tamaño crítico del defecto desde donde 
comenzará la fisura. Existen pocos datos fiables experimentales de la tenacidad 
de la interfaz metal/óxido. La alternativa que utilizamos se basa en un cálculo 
teórico a escala atomística utilizando la Teoría de la Funcional Densidad (DFT) 
\cite{KohnSham65,HohenbergKohn64}. La 
investigación teórica de estas interfaces podría dar condiciones de preferencia de 
diseño de la superficie metálica para disminuir la probabilidad de falla.

\section{Objetivos del Trabajo}

Como requerimiento en el estudio de las interfaces, es necesario conocer las 
propiedades superficiales de los óxidos. 
El objetivo principal del trabajo es calcular, dentro del marco de la DFT
las energías de fractura de las distintas fases del 
ZrO\textsubscript{2}, en las distintas orientaciones cristalinas posibles.

\subsection{Caracterización del estado de equilibrio del ZrO\textsubscript{2}}

El ZrO\textsubscript{2} es estable con la estructura monoclínica (m-ZrO\textsubscript{2 })
a baja temperatura. A temperaturas intermedias, se estabiliza la fase tetragonal 
(t-ZrO\textsubscript{2}), pero en los óxidos protectores esta última puede estar presente
en cantidades considerables \cite{Li2007} ya que puede estabilizarse por tensiones
\cite{Motta2011}. A altas temperaturas, la fase estable es la cúbica (c-ZrO\textsubscript{2}).
Para tomar contacto con las herramientas, se propone estudiar las fases en volumen 
teniendo la optimización de los cristales y sus propiedades elásticas y electrónicas.

\begin{figure}
  \center
  \includegraphics[width=0.25\textwidth]{Figuras/mZrO2_mp-2858_conventional_standard.pdf}
  \includegraphics[width=0.25\textwidth]{Figuras/tZrO2_mp-2574_conventional_standard.pdf}
  \includegraphics[width=0.25\textwidth]{Figuras/cZrO2_mp-1565_conventional_standard.pdf}

  \caption{\protect \label{FiguraCristalesZrO2}
  Estructuras Cristalinas del ZrO\textsubscript{2}, monoclínica (izquierda), tetragonal (centro)
  y cúbica (derecha)
  }
\end{figure}

\subsection{ Cálculo de las energías totales de los bloques de ZrO\textsubscript{2}.}

Las propiedades superficiales de la ZrO2 han sido en parte estudiadas por
Alonso y colaboradores \cite{Alonso2018}. Por otro lado, Ricca
\cite{ricca2015revealing} realiza un estudio de algunas propiedades de las superficies de 
distintas fases de ZrO\textsubscript{2}. Sin embargo, algunas discrepancias entre las referencias
ameritan extender un poco los estudios. 

Por ejemplo las aleaciones de Zr trabajadas mecánicamente adquieren
cierta textura \cite{Malamud2018,Gloaguen2010} que predispone algunas relaciones de orientación 
con el óxido protector. Por lo tanto es necesario investigar las propiedades de las orientaciones
quizá menos estables termodinámicamente. 

\subsection{Cálculo de las energías de fractura del ZrO\textsubscript{2}}

Las Energías de fractura de un cristal pueden obtenerse mediante un cálculo directo 
a partir de los resultados de energías obtenidas por DFT. La metodología ha sido 
ampliamente aplicada en otros sistemas \cite{Liao2010, Liao2010a, Forti2016}. Sin embargo 
no se han encontrado estudios detallados para las distintas fases del ZrO\textsubscript{2}, 
por lo que se pretende avanzar en esta dirección. De ser posible, deberán estudiarse distintas
orientaciones cristalinas para tener un mapa completo de las resistencias relativas entre
los distintos planos cristalinos de las distintas fases. 

\section{Técnicas de cálculo que se usarán}

Los cálculos serán hechos dentro del marco de la DFT \cite{KohnSham65,
HohenbergKohn64} usando el método 
Projector Augmented Wave\cite{Bloch1994,Kresse1999}
implementado en el código Vienna Ab-initio 
Simulation Package\cite{Hafner2007,Hafner2008}
en la Aproximación del Gradiente Generalizado según la 
parametrización PBE \cite{PBE}. 

Se utilizarán herramientas programadas en bash y/o python para procesamiento de 
texto plano y análisis de resultados. Se utilizarán herramientas de conexión 
remota nativas de unix / linux para acceso a las facilidades de cálculo.

\section{Habilidades que desarrollará el estudiante}

\subsection{ Revisión Bibliográfica}

La revisión bibliográfica será un trabajo constante a lo largo de los 6 meses, 
a través de la cual se entrenará en métodos de análisis y búsqueda de 
información.

\subsection{ Cálculos DFT}

Adquirirá experiencia en la aplicación de técnicas de cálculo computacional 
basados en la DFT. Al mismo tiempo aprenderá las relaciones entre los 
parámetros involucrados en los modelos y las cantidades observables 
experimentalmente.

Esta experiencia implica el diseño de la estrategia de cálculo, la 
determinación de los parámetros específicos de acuerdo a su significado teórico 
y el análisis constante de los resultados. La toma de decisiones durante el 
desarrollo de la tesis surgirá tanto del análisis de las mediciones como de la 
comparación crítica de los valores propios con los de la literatura.

\subsection{Conocimiento de las propiedades del  ZrO\textsubscript{2}}

Habrá ganado conocimiento específico sobre los temas del proyecto, 
cálculos específicos de superficie mediante DFT, técnicas de análisis
de propiedades estructurales y electrónicas de cristales y superficies. 

\subsection{ Desarrollo profesional}

Se habrá desempeñado profesionalmente en un grupo de trabajo, con la 
oportunidad de participar de discusiones y de fortalecer su sentido crítico y 
su capacidad colaborativa.

\subsection{ Habilidades de comunicación }

Finalmente, se pondrá énfasis en la habilidad de comunicación de los 
resultados, tanto en la identificación de los datos relevantes como en la 
redacción de los informes. 

\section{Cronograma estimado de cada una de las actividades}
 
Las actividades se desarrollarán a lo largo de 6 meses, y se 
incluye un mínimo de 1 mes para escritura y defensa del Trabajo Final de Ingeniería.

\begin{table}[H]\footnotesize
  \center
  \rowcolors{4}{gray!20!white}{white!20!gray}
  \renewcommand{\arraystretch}{2}
  
  \begin{tabular}{ | p{3cm} | *{6}{m{0.5cm}} | } %c{2cm}|c{2cm}|c{2cm}|c{2cm}|c{2cm}| }
  \hline
    \multicolumn{1}{|c|}{Actividad  } & \multicolumn{6}{c|}{ Mes } \\ [0.5em]
   \cline{2-7}
&1&2&3&4&5&6\\
\hline

\hline
   Revisión bibliográfica & \xmark & \xmark & \xmark & \xmark & \xmark &\xmark \\
   Entrenamiento en el código de cálculo& \xmark & \xmark & \xmark &&&\\
   Cálculo del ZrO2 en volumen &\xmark & \xmark & \xmark &&&\\
   Cálculos de superficie& &&&\xmark & \xmark & \xmark\\
   Análisis y comunicación de resultados && \xmark & \xmark & \xmark & \xmark & \xmark \\
%% \end{tabularx}
  \hline
\end{tabular}
\end{table}

\section{ Materiales e Infraestructura con que se cuenta }

El grupo cuenta con licencia para el uso del código de cálculo VASP\cite{Hafner2007,
Hafner2008} , y con 
2 conjuntos de computadoras (clusters), de 32 procesadores cada uno, para 
cálculo de alto rendimiento utilizados en exclusividad por el grupo. Además, se 
cuenta con acceso a las facilidades de cálculo de altas prestaciones de la CNEA, 
que actualmente cuenta con el cluster ISAAC 750 procesadores y el nuevo cluster 
Neurus de infraestructura dinámica.

\section{Experiencia previa del grupo de trabajo en el tema propuesto Aplicaciones 
de DFT}

En la actualidad la División de Aleaciones Especiales trabaja en el marco de un 
PICT 2018. El mismo implica por un lado
estudios teóricos de los sistemas Zr metálico y de los defectos puntuales de la 
ZrO\textsubscript{2} y la interacción entre ambos sistemas, pero por otro lado se realizan 
estudios experimentales para el diseño de nuevas aleaciones de Zr para uso en 
reactores nucleares de potencia de cuarta generación. 
Por otro lado, los directores poseen amplia experiencia en el estudio de 
sistemas interfaciales mediante métodos computacionales. Sus trabajos 
se llevaron a cabo estudios de adhesión, defectos puntuales y 
difusión que dieron también lugar a publicaciones en revistas internacionales,
tesis de doctorado y trabajos finales de Ingeniería \cite{Cotes2019}. 

\subsection{ Tesis desarrolladas en el grupo en temas de cálculos DFT}

\begin{enumerate}

\item Tesis de Doctorado en Ciencia y Tecnología, Mención Materiales, 
Instituto Sabato, UNSAM: “Modelo atomístico/continuo aplicado a la fractura de 
la capa de óxido en tuberías de reactores nucleares de potencia”. Mariano Forti 
, 2017.

\item Tesis de Doctorado UBA, área Ciencias Físicas: “Defectos 
  constitucionales y energía de migración de aluminio en UAl\textsubscript{4}”. Laura Kniznik, 
2016. 

\end{enumerate}

\subsection{ Publicaciones del grupo en temas de cálculo DFT}

\begin{enumerate}
\item  G.E. Ramírez-Caballero, P.B. Balbuena, P. R. Alonso, P.H. Gargano, G. 
H. Rubiolo, ``Carbon Adsorption and Absorption in the (111) L12 Fe3Al Surface'', J. 
Phys. Chem. C 113 (2009) 18321–18330.

\item  P.R. Alonso, J.R. Fernández, P.H. Gargano, G.H. Rubiolo, U-Al system. 
``Ab-initio and many body potential approaches'', Physica B: Condensed Matter 404 
Issue 18 (2009) 2851-2853.

\item P.R. Alonso, P.H. Gargano, G.E. Ramírez-Caballero, P.B. Balbuena, G.H. 
Rubiolo, ``First principles calculation of L21 + A2 coherent equilibria in the 
Fe-Al-Ti system'', Physica B: Condensed Matter 404 Issue 18 (2009) 2845-2847. 

\item  F. Lanzini, P.H. Gargano, P.R. Alonso, G.H. Rubiolo, ``First principles 
study of phase stabilities in bcc Cu-Al alloy'', Modelling Simul. Mater. Sci. 
Eng. 19 (2011) 015008 (15pp).

\item  P.R. Alonso, P.H. Gargano, P.B. Bozzano, G.E. Ramírez-Caballero, P.B. 
Balbuena, G.H. Rubiolo, ``Combined ab initio and experimental study of A2 + L21 
coherent equilibria in the Fe-Al-X (X=Ti, Nb, V) systems'', Intermetallics 19 
Issue 8 (2011) 1157-1167. 

\item  L. Kniznik, P.R. Alonso, P.H. Gargano, G.H. Rubiolo, ``Simulation of 
UAl4 growth in an UAl3/Al diffusion couple'', Journal of Nuclear Materials 414, 
    309-315. 

\item  P.R. Alonso, P.H. Gargano, G.H. Rubiolo, ``Stability of the C14-laves 
phase (Fe,Si)2Mo from ab initio calculations'', Computer Coupling of Phase 
Diagrams and Thermochemistry (Calphad), 35 (2011) 492–498. 

\item  P.R. Alonso, P.H. Gargano, L. Kniznik, L.M. Pizarro, G.H. Rubiolo; 
``Experimental studies and first principles calculations in nuclear fuel alloys 
for research reactors'' in Nuclear Materials; Editor: Michael P. Hemsworth. 
Series: Physics Research and Technology, Materials Science and Technologies. 
Nova Science Publishers, Inc; 400 Oser Avenue, Suite 1600, Hauppauge, NY 11788, 
EEUU. ISBN: 978-1-61324-010-6. (Nova Science Publishers, Inc , New York, 2011).

\item  P.R. Alonso, P.H. Gargano, G.H. Rubiolo, ``First principles calculation 
of the Al3U-Si3U pseudo binary fcc phase equilibrium diagram'', CALPHAD: Computer 
Coupling of Phase Diagrams and Thermochemistry 38 (2012) 117–121.

\item  M.D. Forti, P.R. Alonso, P.H. Gargano, G.H. Rubiolo, ``Transition 
metals monoxides. An LDA+U study'', Procedia Materials Science 1 (2012) 230 – 234.

\item L. Kniznik, P.R. Alonso, P.H. Gargano, M.D. Forti, G.H. Rubiolo, ``First 
principles study of U-Al system ground state'', Procedia Materials Science 1 , 514 – 519.

\item  M.D. Forti, P. Balbuena, P.R. Alonso, ``Ab-initio studies on 
carburization of Fe3Al based alloys'', Procedia Materials Science 1 (2012) 191 – 
198.

\item  M.D. Forti, P.R. Alonso, P.H. Gargano, G.H. Rubiolo, ``Adhesion Energy 
of the Fe(BCC)/Magnetite Interface within the DFT approach'', Procedia Materials 
Science 8 (2015) 1066 – 1072.

\item  P.R. Alonso, ``Aplicaciones de técnicas de primeros principios al 
cálculo de diagramas de fases de equilibrio''. ``Diagramas de fases de equilibrio 
ternarios Fe-Al-V a partir de expansión en cúmulos y método variacional'', 
conferencista invitada en el ``Workshop en Procesamiento Fïsico Químico 
Avanzado'', Bucaramanga, Colombia, Marzo 2014.

\item  P.R. Alonso, ``Aplicaciones de técnicas de primeros principios al 
cálculo de diagramas de fases de equilibrio. Diagrama de fases de equilibrio 
pseudo-binario UAl3\_USi3 a partir de expansión en cúmulos y simulaciones de 
MonteCarlo'', conferencista invitada en el ''Workshop en Procesamiento Fïsico 
Químico Avanzado'', Bucaramanga, Colombia, Marzo 2014.

\item M.D. Forti, dictado del taller ``VASP'', conferencista invitado en el 
“Workshop en Procesamiento Fïsico Químico Avanzado”, Bucaramanga, Colombia, 
Marzo 2014.

\item  M.D. Forti, ``Propiedades Mecánicas'', conferencista invitado en el 
“Workshop en Procesamiento Fïsico Químico Avanzado”, Bucaramanga, Colombia, 
Marzo 2014.

\item  D. Tozini, M. Forti, P.H. Gargano, Cálculo de diferencias de carga en 
interfaces Fe/Fe3O4 a partir de resultados de DFT, Trabajo 5056, 14º Congreso 
Internacional SAM-CONAMET/IBEROMAT XIII /MATERIA, Santa Fé, 2014.

\item  D. Tozini, M.D. Forti, P.H. Gargano, P.R. Alonso, G.H. Rubiolo, 
Charge difference calculation in Fe/Fe3O4 interfaces from DFT results, Procedia 
Materials Science 9 (2015) 612 – 618.

\item L. Kniznik, P.R. Alonso, P.H. Gargano , G.H. Rubiolo, Energetics and 
electronic structure of UAl4 with point defects, Journal of Nuclear Materials 
466 (2015) 539-550.

\item M.D. Forti, P.R. Alonso, P.H. Gargano, P.B. Balbuena, G.H. Rubiolo, A 
DFT study of atomic structure and adhesion at the Fe(BCC)/Fe3O4 interfaces, 
Surface Science 647 (2016) 55–65.

\item  P.H. Gargano, L. Kniznik, P.R. Alonso, M.D. Forti, G.H. Rubiolo, 
Concentration of constitutional and thermal defects in UAl4, Journal of Nuclear 
Materials 478 (2016) 74-82.
\end{enumerate}

\subsection{ Premios recibidos }

\begin{enumerate} 

\item  Mención especial a Mariano Forti en el CONCURSO ESTÍMULO A JÓVENES 
INVESTIGADORES EN CIENCIA Y TECNOLOGÍA DE MATERIALES. Trabajo premiado: Forti 
M., Balbuena P. y Alonso P., ESTUDIOS AB-INITIO SOBRE CARBURACIÓN EN ALEACIONES 
DE BASE Fe3Al, 11º Congreso Binacional de Metalurgia y Materiales SAM / CONAMET 
2011, 18 al 21 de Octubre de 2011 - Rosario, Argentina.

\item  PREMIO JORGE KITTL. Mejor trabajo en Investigación Básica en Ciencia 
de Materiales. Trabajo premiado: Paula R. Alonso, Pablo H.Gargano y Gerardo H. 
Rubiolo, ESTADO FUNDAMENTAL DEL PSEUDO BINARIO Al3U-Si3U POR PRIMEROS 
PRINCIPIOS. SOLUCIÓN SÓLIDA U(Al,Si)3, 11º Congreso Binacional de Metalurgia y 
Materiales SAM / CONAMET 2011, 18 al 21 de Octubre de 2011 - Rosario, Argentina.

\item  PREMIO JORGE KITTL. Mejor trabajo en Investigación Básica en Ciencia 
de Materiales. Trabajo premiado: Pedro A. Ferreirós, Paula R. Alonso, Pablo H. 
Gargano, Patricia B. Bozzano, Horacio E. Troiani y Gerardo H. Rubiolo, 
    Transformaciones de fase en aleaciones $Fe_{1-2X}Al_X V_X$ (X=1,15), XIII Congreso 
Internacional SAM-CONAMET Iguazú 2013, 20 al 23 de agosto de 2013, Iguazú, 
Misiones, Argentina.

\end{enumerate}

\section{ Proyectos científicos y/o tecnológicos de los que participaría esta tesis  }

\begin{itemize}

\item Los temas a desarrollar en este trabajo son parte de un proyecto
PICT-2018-01671 ``Simulación y ensayo de una aleación propuesta para reactores nucleares de alto quemado''.

\end{itemize}



\printbibliography

\vspace{2cm}
\begin{table}[h!]
  \begin{tabular*}{\textwidth}{ *{3}{>\centering p{0.3\textwidth} } }
    &\\
    \hline
%    \begin{tikzpicture}[overlay, baseline]
%      \node at (1,0.4) {
%      \includegraphics[width=0.3\textwidth]
%      {/home/mariano/Documents/Documentaciones/FirmasMariano.png}
%    };
%      \node at (6,1) {
%      \includegraphics[width=0.3\textwidth]
%      {../Firmas/FirmaKniznik.pdf}
%    };
%    \end{tikzpicture}
    Mariano Forti & Pablo Gargano \\
  \end{tabular*}
\end{table}

