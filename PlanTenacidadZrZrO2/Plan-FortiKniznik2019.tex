\begin{center}

Maestría en Ciencia y Tecnología de Materiales UNSAM / CNEA

Plan de Maestría 2019

\end{center}

\section{ Título de la Tesis }

Modelización Atomística de la tenacidad de interfaces Zr / ZrO2

\section{ Lugar en donde se realizará }
División Aleaciones Especiales, Gerencia Materiales, Gerencia de Área Energía 
Nuclear

\section{Directores}

\subsubsection{Mariano Daniel Forti}

Doctor en Ciencia y Tecnología, Mención Materiales, Instituto Sabato 
UNSAM-CNEA. Investigador CNEA / Docente UNSAM. 
Contacto: \href{mailto:mforti@cnea.gov.ar}{mforti@cnea.gov.ar}, Interno CAC 7832.

\subsubsection{Laura Kniznik}

Doctora UBA, área Ciencias Físicas , Investigadora CNEA / Docente UNSAM y UBA. 
Contacto: \href{mailto:kniznik@cnea.gov.ar}{kniznik@cnea.gov.ar}, Interno CAC: 7832.

\section{Confidencialidad}

La realización de este trabajo no implica un acuerdo de confidencialidad.

\section{Introducción }

El uso de los reactores nucleares como fuente de energía se ha extendido y 
consolidado en la segunda mitad del siglo XX y en los inicios de este siglo 
XXI. La continuidad de este recurso depende en gran medida de que las plantas 
nucleares sean cada vez más seguras y confiables, lo cual implica un 
mejoramiento de todos los aspectos relacionados con su operación. Entre ellos 
se cuenta la integridad de los elementos estructurales, que depende fuertemente 
de la estabilidad mecánica de los óxidos protectores. En las centrales CNA-I y 
CNA-II el Zircaloy-4 (aleación base Zr con adiciones de Sn, Fe and Cr) se 
utiliza en los canales refrigerantes\cite{Estevez2000}. Sin embargo, esta 
aleación presenta 
problemas de corrosión que, por ejemplo, motivaron el reemplazo de los canales 
refrigerantes de CNA-I en 1988.

La capa de óxido que se forma sobre la superficie metálica se comporta como una 
protección contra la absorción de hidrógeno y la degradación. Si la capa de 
óxido se rompe por la formación de fisuras, la película de óxido cercana a la 
superficie se enriquece en hidrógeno favoreciendo la difusión hacia el metal en 
detrimento de su resistencia mecánica. La formación de fisuras puede ocurrir 
debido a la transformación del óxido tetragonal a monoclínico dado que dicha 
transformación involucra cambios de volumen que pueden inducir tensiones 
internas. El objetivo de minimizar los efectos de dicha transformación requiere 
el control de la textura y de la composición de la superficie metálica. Una 
estrategia para evaluar la resistencia de la interfaz metal/óxido, consiste en 
el cálculo del trabajo de separación de dicha interfaz.

El desprendimiento es función del espesor de la capa de óxido\cite{Schutze2005},
pero el modo 
de falla de la capa de óxido depende del espesor de la misma y del estado de 
tensiones del sistema metal-óxido por lo que el conocimiento de las propiedades 
de la interfaz es de vital importancia. Debe conocerse la tenacidad de la 
interfaz Zr/ZrO2 (tetragonal) y el tamaño crítico del defecto desde donde 
comenzará la fisura. Existen pocos datos fiables experimentales de la tenacidad 
de la interfaz metal/óxido. La alternativa que utilizamos se basa en un cálculo 
teórico a escala atomística, la Teoría de la Funcional Densidad (DFT) 
\cite{KohnSham65,HohenbergKohn64}. La 
investigación teórica de esta interfaz podría dar condiciones de preferencia de 
diseño de la superficie metálica para disminuir la probabilidad de falla.

\section{Objetivos del Trabajo}

EL objetivo principal del trabajo es calcular, dentro del marco de la Teoría de 
la Funcional Densidad la tenacidad de la interfaz Zr / ZrO2, procediendo como
se explica a continuación.

\subsection{ Cálculo de las energías totales de los bloques de Zr y de ZrO2.}

El estudio de las superficies es necesario para la construcción del modelo de 
interfaz.

\subsection{ Cálculo de la energía de la interfaz.}

Se analizarán composición y orientaciones de las superficies para elegir la 
configuración más probable.

\subsection{Resistencia de la interfaz.}

 Se analizará la respuesta el sistema ante solicitaciones de tracción para 
obtener el trabajo de separación. La tenacidad de la interfaz metal/óxido se 
obtendrá integrando la relación tracción-separación del óxido respecto del 
metal.

\section{Descripción sucinta de los conocimientos y habilidades que el/la 
maestrando/a adquirirá durante este trabajo }

\subsection{ Revisión Bibliográfica}

La revisión bibliográfica será un trabajo constante a lo largo de los 13 meses, 
a través de la cual se entrenará en métodos de análisis y búsqueda de 
información.

\subsection{7.2  Cálculos DFT}

Adquirirá experiencia en la aplicación de técnicas de cálculo computacional 
basados en la DFT. Al mismo tiempo aprenderá las relaciones entre los 
parámetros involucrados en los modelos y las cantidades observables 
experimentalmente.

Esta experiencia implica el diseño de la estrategia de cálculo, la 
determinación de los parámetros específicos de acuerdo a su significado teórico 
y el análisis constante de los resultados. La toma de decisiones durante el 
desarrollo de la tesis surgirá tanto del análisis de las mediciones como de la 
comparación crítica de los valores propios con los de la literatura.

\subsection{ Conocimiento del sistema interfacial Zr/ZrO2 }

Habrá ganado conocimiento específico sobre los temas del proyecto, 
microestructura y tenacidad de la interfaz metal / óxido metálico. 

\subsection{7.4  Desarrollo profesional}

Se habrá desempeñado profesionalmente en un grupo de trabajo, con la 
oportunidad de participar de discusiones y de fortalecer su sentido crítico y 
su capacidad colaborativa.

\subsection{ Habilidades de comunicación }

Finalmente, se pondrá énfasis en la habilidad de comunicación de los 
resultados, tanto en la identificación de los datos relevantes como en la 
redacción de los informes. 

\section{Trabajos a realizar}

\subsection{ Revisión Bibliográfica. }

De esta revisión deben decidirse la relación de orientaciones más probable 
entre los constituyentes de la interfaz. Por otro lado debe hacerse un recuento 
de las áreas de incidencia tecnológica del problema. También se deberá hacer 
una búsqueda constante durante todo el periodo sobre nuevos datos 
experimentales de adhesión en el sistema de interés

\subsection{ Construcción del modelo atomístico de la interfaz.}

A partir de las relaciones de orientaciones obtenida de la revisión 
bibliográfica, se construirán los modelos de interfaz para las posibles 
configuraciones de apilamiento a nivel atomístico, entre las posibilidades que 
no se puedan discernir experimentalmente. Se construirán los bloques 
superficiales en función de estos datos haciendo un estudio de convergencia de 
las propiedades superficiales en función del número de capas atómicas y los 
parámetros de cálculo. Con los bloques superficiales se armará por apilamiento 
los modelos de interfaz. Utilizando los resultados de los cálculos de energía 
en función de la separación interfacial para las distintas configuraciones 
\cite{Jiangetal} 
se decidirá la interfaz más adherida.

\section{Técnicas experimentales o de cálculo que se usarán}

Los cálculos serán hechos dentro del marco de la DFT \cite{KohnSham65,
HohenbergKohn64} usando el método 
Projector Augmented Wave\cite{Bloch1994,Kresse1999}
implementado en el código Vienna Ab-initio 
Simulation Package\cite{Hafner2007,Hafner2008}
en la Aproximación del Gradiente Generalizado según la 
parametrización PBE \cite{PBE}. 

Se utilizarán herramientas programadas en bash y/o python para procesamiento de 
texto plano y análisis de resultados. Se utilizarán herramientas de conección 
remota nativas de unix / linux para acceso a las facilidades de cálculo.

\section{Cronograma estimado de cada una de las actividades}

 (deben totalizar 13 
meses incluido un mínimo de 3 meses para escritura y defensa) 


Actividad
Meses

%\begin{table}
%
%0   & Revisión bibliográfica &Entrenamiento en el código de cálculo&Cálculo de los bloques relajados de metal y óxido por separado&Cálculo de la energía de la interfaz metal/óxido en función de la separación &Análisis de resultados y escritura de la Tesis
%1  & X                      &X                                    &                                                              &                                                                             &
%2  & X                      &X                                    &                                                              &                                                                             &
%3  &                        &X                                    &X                                                             &                                                                             &
%4  &                        &                                     &X                                                             &                                                                             &X
%5  &                        &                                     &X                                                             &X                                                                            &X
%6  &                        &                                     &                                                              &X                                                                            &X
%7  & X                      &                                     &                                                              &X                                                                            &X
%8  & X                      &                                     &                                                              &X                                                                            &X
%9  & X                      &                                     &                                                              &X                                                                            &X
%10 &                        &                                     &                                                              &X                                                                            &X
%11 & X                      &                                     &                                                              &                                                                             &X
%12 &                        &                                     &                                                              &                                                                             &X
%13 &                        &                                     &                                                              &                                                                              X
%\end{table}

\section{ Materiales e Infraestructura con que se cuenta }

El grupo cuenta con licencia para el uso del código de cálculo VASP\cite{Hafner2007,
Hafner2008} , y con 
2 conjuntos de computadoras (clusters), de 32 procesadores cada uno, para 
cálculo de alto rendimiento utilizados en exclusividad por el grupo. Además, se 
cuenta con acceso a las facilidades de cálculo de alta performance de la CNEA, 
que actualmente cuenta con el cluster ISAAC 750 procesadores y el nuevo cluster 
Neurus de 1040 procesadores.

\section{Experiencia previa del grupo de trabajo en el tema propuesto Aplicaciones 
de DFT}

En la actualidad la División de Aleaciones Especiales trabaja en el marco de un 
PICT 2015 y un Proyecto UNSAM 2017-2019. Estos proyectos implican por un lado 
estudios teóricos de los sistemas Zr metálico y de los defectos puntuales de la 
ZrO2 y la interacción entre ambos sistemas, pero por otro lado se realizan 
estudios experimentales para el diseño de nuevas aleaciones de Zr para uso en 
reactores nucleares de potencia de cuarta generación. 
Por otro lado, los directores poseen amplia experiencia en el estudio de 
sistemas interfaciales mediante métodos computacionales. En sus tesis de 
doctorado se llevaron a cabo estudios de adhesión, defectos puntuales y 
difusión que dieron también lugar a publicaciones en revistas internacionales. 

\subsection{ Tesis desarrolladas en el grupo en temas de cálculos DFT}

\begin{enumerate}


\item Tesis de Doctorado en Ciencia y Tecnología, Mención Materiales, 
Instituto Sabato, UNSAM: “Modelo atomístico/continuo aplicado a la fractura de 
la capa de óxido en tuberías de reactores nucleares de potencia”. Mariano Forti 
, 2017.

\item Tesis de Doctorado UBA, área Ciencias Físicas: “Defectos 
constitucionales y energía de migración de aluminio en UAl4”. Laura Kniznik, 
2016. 

\end{enumerate}

\subsection{ Publicaciones del grupo en temas de cálculo DFT}

\begin{enumerate}
\item  G.E. Ramírez-Caballero, P.B. Balbuena, P. R. Alonso, P.H. Gargano, G. 
H. Rubiolo, Carbon Adsorption and Absorption in the (111) L12 Fe3Al Surface, J. 
Phys. Chem. C 113 (2009) 18321–18330.

\item  P.R. Alonso, J.R. Fernández, P.H. Gargano, G.H. Rubiolo, U-Al system. 
Ab-initio and many body potential approaches, Physica B: Condensed Matter 404 
Issue 18 (2009) 2851-2853.

\item P.R. Alonso, P.H. Gargano, G.E. Ramírez-Caballero, P.B. Balbuena, G.H. 
Rubiolo, First principles calculation of L21 + A2 coherent equilibria in the 
Fe-Al-Ti system, Physica B: Condensed Matter 404 Issue 18 (2009) 2845-2847. 

\item  F. Lanzini, P.H. Gargano, P.R. Alonso, G.H. Rubiolo, First principles 
study of phase stabilities in bcc Cu-Al alloy, Modelling Simul. Mater. Sci. 
Eng. 19 (2011) 015008 (15pp).

\item  P.R. Alonso, P.H. Gargano, P.B. Bozzano, G.E. Ramírez-Caballero, P.B. 
Balbuena, G.H. Rubiolo, Combined ab initio and experimental study of A2 + L21 
coherent equilibria in the Fe-Al-X (X=Ti, Nb, V) systems, Intermetallics 19 
Issue 8 (2011) 1157-1167. 

\item  L. Kniznik, P.R. Alonso, P.H. Gargano, G.H. Rubiolo, Simulation of 
UAl4 growth in an UAl3/Al diffusion couple, Journal of Nuclear Materials 414 

\item 309-315. 

\item  P.R. Alonso, P.H. Gargano, G.H. Rubiolo, Stability of the C14-laves 
phase (Fe,Si)2Mo from ab initio calculations, Computer Coupling of Phase 
Diagrams and Thermochemistry (Calphad), 35 (2011) 492–498. 

\item  P.R. Alonso, P.H. Gargano, L. Kniznik, L.M. Pizarro, G.H. Rubiolo; 
"Experimental studies and first principles calculations in nuclear fuel alloys 
for research reactors", in Nuclear Materials; Editor: Michael P. Hemsworth. 
Series: Physics Research and Technology, Materials Science and Technologies. 
Nova Science Publishers, Inc; 400 Oser Avenue, Suite 1600, Hauppauge, NY 11788, 
EEUU. ISBN: 978-1-61324-010-6. (Nova Science Publishers, Inc , New York, 2011).

\item  P.R. Alonso, P.H. Gargano, G.H. Rubiolo, First principles calculation 
of the Al3U-Si3U pseudo binary fcc phase equilibrium diagram, CALPHAD: Computer 
Coupling of Phase Diagrams and Thermochemistry 38 (2012) 117–121.

\item  M.D. Forti, P.R. Alonso, P.H. Gargano, G.H. Rubiolo, Transition 
metals monoxides. An LDA+U study, Procedia Materials Science 1 (2012) 230 – 234.

\item L. Kniznik, P.R. Alonso, P.H. Gargano, M.D. Forti, G.H. Rubiolo, First 
principles study of U-Al system ground state, Procedia Materials Science 1 , 514 – 519.

\item  M.D. Forti, P. Balbuena, P.R. Alonso, Ab-initio studies on 
carburization of Fe3Al based alloys, Procedia Materials Science 1 (2012) 191 – 
198.

\item  M.D. Forti, P.R. Alonso, P.H. Gargano, G.H. Rubiolo, Adhesion Energy 
of the Fe(BCC)/Magnetite Interface within the DFT approach, Procedia Materials 
Science 8 (2015) 1066 – 1072.

\item  P.R. Alonso, “Aplicaciones de técnicas de primeros principios al 
cálculo de diagramas de fases de equilibrio. Diagramas de fases de equilibrio 
ternarios Fe-Al-V a partir de expansión en cúmulos y método variacional”, 
conferencista invitada en el “Workshop en Procesamiento Fïsico Químico 
Avanzado”, Bucaramanga, Colombia, Marzo 2014.

\item  P.R. Alonso, “Aplicaciones de técnicas de primeros principios al 
cálculo de diagramas de fases de equilibrio. Diagrama de fases de equilibrio 
pseudo-binario UAl3\_USi3 a partir de expansión en cúmulos y simulaciones de 
MonteCarlo”, conferencista invitada en el “Workshop en Procesamiento Fïsico 
Químico Avanzado”, Bucaramanga, Colombia, Marzo 2014.

\item M.D. Forti, dictado del taller "VASP", conferencista invitado en el 
“Workshop en Procesamiento Fïsico Químico Avanzado”, Bucaramanga, Colombia, 
Marzo 2014.

\item  M.D. Forti, "Propiedades Mecánicas", conferencista invitado en el 
“Workshop en Procesamiento Fïsico Químico Avanzado”, Bucaramanga, Colombia, 
Marzo 2014.

\item  D. Tozini, M. Forti, P.H. Gargano, Cálculo de diferencias de carga en 
interfaces Fe/Fe3O4 a partir de resultados de DFT, Trabajo 5056, 14º Congreso 
Internacional SAM-CONAMET/IBEROMAT XIII /MATERIA, Santa Fé, 2014.

\item  D. Tozini, M.D. Forti, P.H. Gargano, P.R. Alonso, G.H. Rubiolo, 
Charge difference calculation in Fe/Fe3O4 interfaces from DFT results, Procedia 
Materials Science 9 (2015) 612 – 618.

\item L. Kniznik, P.R. Alonso, P.H. Gargano , G.H. Rubiolo, Energetics and 
electronic structure of UAl4 with point defects, Journal of Nuclear Materials 
466 (2015) 539-550.

\item M.D. Forti, P.R. Alonso, P.H. Gargano, P.B. Balbuena, G.H. Rubiolo, A 
DFT study of atomic structure and adhesion at the Fe(BCC)/Fe3O4 interfaces, 
Surface Science 647 (2016) 55–65.

\item  P.H. Gargano, L. Kniznik, P.R. Alonso, M.D. Forti, G.H. Rubiolo, 
Concentration of constitutional and thermal defects in UAl4, Journal of Nuclear 
Materials 478 (2016) 74-82.
\end{enumerate}

\subsection{ Premios recibidos }

\begin{enumerate} 


\item  Mención especial a Mariano Forti en el CONCURSO ESTÍMULO A JÓVENES 
INVESTIGADORES EN CIENCIA Y TECNOLOGÍA DE MATERIALES. Trabajo premiado: Forti 
M., Balbuena P. y Alonso P., ESTUDIOS AB-INITIO SOBRE CARBURACIÓN EN ALEACIONES 
DE BASE Fe3Al, 11º Congreso Binacional de Metalurgia y Materiales SAM / CONAMET 
2011, 18 al 21 de Octubre de 2011 - Rosario, Argentina.

\item  PREMIO JORGE KITTL. Mejor trabajo en Investigación Básica en Ciencia 
de Materiales. Trabajo premiado: Paula R. Alonso, Pablo H.Gargano y Gerardo H. 
Rubiolo, ESTADO FUNDAMENTAL DEL PSEUDO BINARIO Al3U-Si3U POR PRIMEROS 
PRINCIPIOS. SOLUCIÓN SÓLIDA U(Al,Si)3, 11º Congreso Binacional de Metalurgia y 
Materiales SAM / CONAMET 2011, 18 al 21 de Octubre de 2011 - Rosario, Argentina.

\item  PREMIO JORGE KITTL. Mejor trabajo en Investigación Básica en Ciencia 
de Materiales. Trabajo premiado: Pedro A. Ferreirós, Paula R. Alonso, Pablo H. 
Gargano, Patricia B. Bozzano, Horacio E. Troiani y Gerardo H. Rubiolo, 
    Transformaciones de fase en aleaciones $Fe_{1-2X}Al_X V_X$ (X=1,15), XIII Congreso 
Internacional SAM-CONAMET Iguazú 2013, 20 al 23 de agosto de 2013, Iguazú, 
Misiones, Argentina.

\end{enumerate}

\subsection{ Proyectos científicos y/o tecnológicos de los que participaría esta tesis  }

\begin{enumerate}


\item “Buscando una nueva aleación para el elemento combustible CAREM-25” – 
ANPCyT, PICT-2015-2267. Titular: Gerardo H. Rubiolo. Grupo Responsable: Paula 
R. Alonso, Pablo H. Gargano, Liliana A. Lanzani. Grupo Colaborador: Laura 
Kniznik, Pedro A. Ferreirós, Mariano D. Forti, Héctor A. Raffaelli, Sergio V. 
Ilarri, Mario A. Acosta, Perla B. Balbuena. Fecha de vigencia: 2016 – 2019.

\item “Buscando una nueva aleación para el elemento combustible CAREM-25” - 
Proyecto UNSAM 80020160500046SM con vigencia Enero 2017-Diciembre 2019. 
Directora del proyecto: Paula Regina Alonso.

\end{enumerate}

% Referencias
% [1]	Estévez E, M. M, G. R, Fuel element mechanical design for CAREM-25 
% reactor, International Nuclear Information System. 32 (2000) 32046443.
% [2]	M. Schütze, Modelling oxide scale fracture, Materials at High 
% Temperatures. 22 (2005) 147–154. doi:10.3184/096034005782750437.
% [3]	W. Kohn, L.J. Sham, Self-Consistent Equations Including Exchange and 
% Correlation Effects, Physical Review. 140 (1965) A1133–A1138. 
% doi:10.1103/PhysRev.140.A1133.
% [4]	P. Hohenberg, W. Kohn, Inhomogeneous electron gas, Phys. Rev. 136 
% (1964) B864–B871.
% [5]	Y. Jiang, Y. Wei, J.R. Smith, J.W. Hutchinson, A.G. Evans, First 
% principles based predictions of the toughness of a metal/oxide interface, 
% International Journal of Materials Research (Formerly Zeitschrift Fuer 
% Metallkunde). 101 (2010) 8–15. doi:10.3139/146.110254.
% [6]	P.E. Blöch, Projector augmented-wave method, Physical Review B. 50 
% (1994) 13063–17979. doi:10.1103/PhysRevB.50.17953.
% [7]	G. Kresse, D. Joubert, From ultrasoft pseudopotentials to the projector 
% augmented-wave method, Physical Review B. 59 (1999) 1758–1775. 
% doi:10.1103/PhysRevB.59.1758.
% [8]	J. Hafner, Ab-initio simulations of materials using VASP: 
% Density-functional theory and beyond, Journal of Computational Chemistry. 29 
% (2008) 2044–2078. doi:10.1002/jcc.
% [9]	J. Hafner, Materials simulations using VASP—a quantum perspective to 
% materials science, Computer Physics Communications. 177 (2007) 6–13. 
% doi:10.1016/j.cpc.2007.02.045.
% [10]	J. Perdew, K. Burke, M. Ernzerhof, Generalized Gradient Approximation 
% Made Simple., Physical Review Letters. 77 (1996) 3865–3868.



\bibliography{Bibliography}
\bibliographystyle{ieeetr}










Mariano Forti
Laura Kniznik

