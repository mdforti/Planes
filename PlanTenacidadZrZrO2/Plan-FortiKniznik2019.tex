
Maestría en Ciencia y Tecnología de Materiales UNSAM / CNEA
Plan de Maestría 2019

\section{ Título de la Tesis }

Modelización Atomística de la tenacidad de interfaces Zr / ZrO2

\section{ Lugar en donde se realizará }
División Aleaciones Especiales, Gerencia Materiales, Gerencia de Área Energía Nu
clear

\section{Directores}

\subsection{Mariano Daniel Forti}

Doctor en Ciencia y Tecnología, Mención Materiales, Instituto Sabato UNSAM-CNEA.
 Investigador CNEA / Docente UNSAM. 

Contacto: mforti@cnea.gov.ar, Interno CAC 7832.

\subsection{Laura Kniznik}

Doctora UBA, área Ciencias Físicas , Investigadora CNEA / Docente UNSAM y UBA. C
ontacto: kniznik@cnea.gov.ar, Interno CAC: 7832.

\subsection{Confidencialidad}

La realización de este trabajo no implica un acuerdo de confidencialidad.

\section{Introducción }

El uso de los reactores nucleares como fuente de energía se ha extendido y conso
lidado en la segunda mitad del siglo XX y en los inicios de este siglo XXI. La c
ontinuidad de este recurso depende en gran medida de que las plantas nucleares 
sean cada vez más seguras y confiables, lo cual implica un mejoramiento de todos 
los aspectos relacionados con su operación. Entre ellos se cuenta la integridad de los elementos estructurales, que depende fuertemente de la estabilidad mecáni
ca de los óxidos protectores. En las centrales CNA-I y CNA-II el Zircaloy-4 (ale
ación base Zr con adiciones de Sn, Fe and Cr) se utiliza en los canales refriger
antes [1]. Sin embargo, esta aleación presenta problemas de corrosión que, por e
jemplo, motivaron el reemplazo de los canales refrigerantes de CNA-I en 1988.
La capa de óxido que se forma sobre la superficie metálica se comporta como una 
protección contra la absorción de hidrógeno y la degradación. Si la capa de óxid
o se rompe por la formación de fisuras, la película de óxido cercana a la superf
icie se enriquece en hidrógeno favoreciendo la difusión hacia el metal en detrim
ento de su resistencia mecánica. La formación de fisuras puede ocurrir debido a 
la transformación del óxido tetragonal a monoclínico dado que dicha transformaci
ón involucra cambios de volumen que pueden inducir tensiones internas. El objeti
vo de minimizar los efectos de dicha transformación requiere el control de la te
xtura y de la composición de la superficie metálica. Una estrategia para evaluar
 la resistencia de la interfaz metal/óxido, consiste en el cálculo del trabajo d
e separación de dicha interfaz.
El desprendimiento es función del espesor de la capa de óxido[2], pero el modo d
e falla de la capa de óxido depende del espesor de la misma y del estado de tens
iones del sistema metal-óxido por lo que el conocimiento de las propiedades de l
a interfaz es de vital importancia. Debe conocerse la tenacidad de la interfaz Z
r/ZrO2 (tetragonal) y el tamaño crítico del defecto desde donde comenzará la fis
ura. Existen pocos datos fiables experimentales de la tenacidad de la interfaz m
etal/óxido. La alternativa que utilizamos se basa en un cálculo teórico a escala
 atomística, la Teoría de la Funcional Densidad (DFT) [3,4]. La investigación te
órica de esta interfaz podría dar condiciones de preferencia de diseño de la sup
erficie metálica para disminuir la probabilidad de falla.
6. Objetivos del Trabajo
EL objetivo principal del trabajo es calcular, dentro del marco de la Teoría de 
la Funcional Densidad la tenacidad de la interfaz Zr / ZrO2, con los siguientes 
pasos:
6.1  Cálculo de las energías totales de los bloques de Zr y de ZrO2.
El estudio de las superficies es necesario para la construcción del modelo de in
terfaz.
6.2  Cálculo de la energía de la interfaz.
Se analizarán composición y orientaciones de las superficies para elegir la conf
iguración más probable.
6.3  Resistencia de la interfaz.
 Se analizará la respuesta el sistema ante solicitaciones de tracción para obten
er el trabajo de separación. La tenacidad de la interfaz metal/óxido se obtendrá
 integrando la relación tracción-separación del óxido respecto del metal.
7. Descripción sucinta de los conocimientos y habilidades que el/la maestrando/a
 adquirirá durante este trabajo
7.1  Revisión Bibliográfica
La revisión bibliográfica será un trabajo constante a lo largo de los 13 meses, 
a través de la cual se entrenará en métodos de análisis y búsqueda de informació
n.
7.2  Cálculos DFT
Adquirirá experiencia en la aplicación de técnicas de cálculo computacional basa
dos en la DFT. Al mismo tiempo aprenderá las relaciones entre los parámetros inv
olucrados en los modelos y las cantidades observables experimentalmente.
Esta experiencia implica el diseño de la estrategia de cálculo, la determinación
 de los parámetros específicos de acuerdo a su significado teórico y el análisis
 constante de los resultados. La toma de decisiones durante el desarrollo de la 
tesis surgirá tanto del análisis de las mediciones como de la comparación crític
a de los valores propios con los de la literatura.
7.3  Conocimiento del sistema interfacial Zr/ZrO2
Habrá ganado conocimiento específico sobre los temas del proyecto, microestructu
ra y tenacidad de la interfaz metal / óxido metálico. 
7.4  Desarrollo profesional
Se habrá desempeñado profesionalmente en un grupo de trabajo, con la oportunidad
 de participar de discusiones y de fortalecer su sentido crítico y su capacidad 
colaborativa.
7.5  Habilidades de comunicación
Finalmente, se pondrá énfasis en la habilidad de comunicación de los resultados,
 tanto en la identificación de los datos relevantes como en la redacción de los 
informes. 
8. Trabajos a realizar
8.1  Revisión Bibliográfica. 
De esta revisión deben decidirse la relación de orientaciones más probable entre
 los constituyentes de la interfaz. Por otro lado debe hacerse un recuento de la
s áreas de incidencia tecnológica del problema. También se deberá hacer una búsq
ueda constante durante todo el periodo sobre nuevos datos experimentales de adhe
sión en el sistema de interés
8.2  Construcción del modelo atomístico de la interfaz.
A partir de las relaciones de orientaciones obtenida de la revisión bibliográfic
a, se construirán los modelos de interfaz para las posibles configuraciones de a
pilamiento a nivel atomístico, entre las posibilidades que no se puedan discerni
r experimentalmente. Se construirán los bloques superficiales en función de esto
s datos haciendo un estudio de convergencia de las propiedades superficiales en 
función del número de capas atómicas y los parámetros de cálculo. Con los bloque
s superficiales se armará por apilamiento los modelos de interfaz. Utilizando lo
s resultados de los cálculos de energía en función de la separación interfacial 
para las distintas configuraciones [5] se decidirá la interfaz más adherida.
9. Técnicas experimentales o de cálculo que se usarán 
Los cálculos serán hechos dentro del marco de la DFT[3,4] usando el método Proje
ctor Augmented Wave[6,7] implementado en el código Vienna Ab-initio Simulation P
ackage[8,9] en la Aproximación del Gradiente Generalizado según la parametrizaci
ón PBE [10]. 
Se utilizarán herramientas programadas en bash y/o python para procesamiento de 
texto plano y análisis de resultados. Se utilizarán herramientas de conección re
mota nativas de unix / linux para acceso a las facilidades de cálculo.
10. Cronograma estimado de cada una de las actividades (deben totalizar 13 meses
 incluido un mínimo de 3 meses para escritura y defensa) 
Actividad
Meses

1
2
3
4
5
6
7
8
9
10
11
12
13
Revisión bibliográfica 
X
X




X
X
X

X


Entrenamiento en el código de cálculo
X
X
X










Cálculo de los bloques relajados de metal y óxido por separado


X
X
X








Cálculo de la energía de la interfaz metal/óxido en función de la separación




X
X
X
X
X
X



Análisis de resultados y escritura de la Tesis



X
X
X
X
X
X
X
X
X
X
11. Materiales e Infraestructura con que se cuenta 
El grupo cuenta con licencia para el uso del código de cálculo VASP[8,9], y con 
2 conjuntos de computadoras (clusters), de 32 procesadores cada uno, para cálcul
o de alto rendimiento utilizados en exclusividad por el grupo. Además, se cuenta
 con acceso a las facilidades de cálculo de alta performance de la CNEA, que act
ualmente cuenta con el cluster ISAAC 750 procesadores y el nuevo cluster Neurus 
de 1040 procesadores.
12. Experiencia previa del grupo de trabajo en el tema propuesto Aplicaciones de
 DFT
En la actualidad la División de Aleaciones Especiales trabaja en el marco de un 
PICT 2015 y un Proyecto UNSAM 2017-2019. Estos proyectos implican por un lado es
tudios teóricos de los sistemas Zr metálico y de los defectos puntuales de la Zr
O2 y la interacción entre ambos sistemas, pero por otro lado se realizan estudio
s experimentales para el diseño de nuevas aleaciones de Zr para uso en reactores
 nucleares de potencia de cuarta generación. 
Por otro lado, los directores poseen amplia experiencia en el estudio de sistema
s interfaciales mediante métodos computacionales. En sus tesis de doctorado se l
levaron a cabo estudios de adhesión, defectos puntuales y difusión que dieron ta
mbién lugar a publicaciones en revistas internacionales. 
12.1  Tesis desarrolladas en el grupo en temas de cálculos DFT
    (1) Tesis de Doctorado en Ciencia y Tecnología, Mención Materiales, Institut
o Sabato, UNSAM: “Modelo atomístico/continuo aplicado a la fractura de la capa d
e óxido en tuberías de reactores nucleares de potencia”. Mariano Forti , 2017.
    (2) Tesis de Doctorado UBA, área Ciencias Físicas: “Defectos constitucionale
s y energía de migración de aluminio en UAl4”. Laura Kniznik, 2016. 
12.2  Publicaciones del grupo en temas de cálculo DFT
    (1)  G.E. Ramírez-Caballero, P.B. Balbuena, P. R. Alonso, P.H. Gargano, G. H
. Rubiolo, Carbon Adsorption and Absorption in the (111) L12 Fe3Al Surface, J. P
hys. Chem. C 113 (2009) 18321–18330.
    (2)  P.R. Alonso, J.R. Fernández, P.H. Gargano, G.H. Rubiolo, U-Al system. A
b-initio and many body potential approaches, Physica B: Condensed Matter 404 Iss
ue 18 (2009) 2851-2853.
    (3) P.R. Alonso, P.H. Gargano, G.E. Ramírez-Caballero, P.B. Balbuena, G.H. R
ubiolo, First principles calculation of L21 + A2 coherent equilibria in the Fe-A
l-Ti system, Physica B: Condensed Matter 404 Issue 18 (2009) 2845-2847. 
    (4)  F. Lanzini, P.H. Gargano, P.R. Alonso, G.H. Rubiolo, First principles s
tudy of phase stabilities in bcc Cu-Al alloy, Modelling Simul. Mater. Sci. Eng. 
19 (2011) 015008 (15pp).
    (5)  P.R. Alonso, P.H. Gargano, P.B. Bozzano, G.E. Ramírez-Caballero, P.B. B
albuena, G.H. Rubiolo, Combined ab initio and experimental study of A2 + L21 coh
erent equilibria in the Fe-Al-X (X=Ti, Nb, V) systems, Intermetallics 19 Issue 8
 (2011) 1157-1167. 
    (6)  L. Kniznik, P.R. Alonso, P.H. Gargano, G.H. Rubiolo, Simulation of UAl4
 growth in an UAl3/Al diffusion couple, Journal of Nuclear Materials 414 (2011) 
309-315. 
    (7)  P.R. Alonso, P.H. Gargano, G.H. Rubiolo, Stability of the C14-laves pha
se (Fe,Si)2Mo from ab initio calculations, Computer Coupling of Phase Diagrams a
nd Thermochemistry (Calphad), 35 (2011) 492–498. 
    (8)  P.R. Alonso, P.H. Gargano, L. Kniznik, L.M. Pizarro, G.H. Rubiolo; "Exp
erimental studies and first principles calculations in nuclear fuel alloys for r
esearch reactors", in Nuclear Materials; Editor: Michael P. Hemsworth. Series: P
hysics Research and Technology, Materials Science and Technologies. Nova Science
 Publishers, Inc; 400 Oser Avenue, Suite 1600, Hauppauge, NY 11788, EEUU. ISBN: 
978-1-61324-010-6. (Nova Science Publishers, Inc , New York, 2011).
    (9)  P.R. Alonso, P.H. Gargano, G.H. Rubiolo, First principles calculation o
f the Al3U-Si3U pseudo binary fcc phase equilibrium diagram, CALPHAD: Computer C
oupling of Phase Diagrams and Thermochemistry 38 (2012) 117–121.
    (10)  M.D. Forti, P.R. Alonso, P.H. Gargano, G.H. Rubiolo, Transition metals
 monoxides. An LDA+U study, Procedia Materials Science 1 (2012) 230 – 234.
    (11) L. Kniznik, P.R. Alonso, P.H. Gargano, M.D. Forti, G.H. Rubiolo, First 
principles study of U-Al system ground state, Procedia Materials Science 1 (2012
) 514 – 519.
    (12)  M.D. Forti, P. Balbuena, P.R. Alonso, Ab-initio studies on carburizati
on of Fe3Al based alloys, Procedia Materials Science 1 (2012) 191 – 198.
    (13)  M.D. Forti, P.R. Alonso, P.H. Gargano, G.H. Rubiolo, Adhesion Energy o
f the Fe(BCC)/Magnetite Interface within the DFT approach, Procedia Materials Sc
ience 8 (2015) 1066 – 1072.
    (14)  P.R. Alonso, “Aplicaciones de técnicas de primeros principios al cálcu
lo de diagramas de fases de equilibrio. Diagramas de fases de equilibrio ternari
os Fe-Al-V a partir de expansión en cúmulos y método variacional”, conferencista
 invitada en el “Workshop en Procesamiento Fïsico Químico Avanzado”, Bucaramanga
, Colombia, Marzo 2014.
    (15)  P.R. Alonso, “Aplicaciones de técnicas de primeros principios al cálcu
lo de diagramas de fases de equilibrio. Diagrama de fases de equilibrio pseudo-b
inario UAl3_USi3 a partir de expansión en cúmulos y simulaciones de MonteCarlo”,
 conferencista invitada en el “Workshop en Procesamiento Fïsico Químico Avanzado
”, Bucaramanga, Colombia, Marzo 2014.
    (16) M.D. Forti, dictado del taller "VASP", conferencista invitado en el “Wo
rkshop en Procesamiento Fïsico Químico Avanzado”, Bucaramanga, Colombia, Marzo 2
014.
    (17)  M.D. Forti, "Propiedades Mecánicas", conferencista invitado en el “Wor
kshop en Procesamiento Fïsico Químico Avanzado”, Bucaramanga, Colombia, Marzo 20
14.
    (18)  D. Tozini, M. Forti, P.H. Gargano, Cálculo de diferencias de carga en 
interfaces Fe/Fe3O4 a partir de resultados de DFT, Trabajo 5056, 14º Congreso In
ternacional SAM-CONAMET/IBEROMAT XIII /MATERIA, Santa Fé, 2014.
    (19)  D. Tozini, M.D. Forti, P.H. Gargano, P.R. Alonso, G.H. Rubiolo, Charge
 difference calculation in Fe/Fe3O4 interfaces from DFT results, Procedia Materi
als Science 9 (2015) 612 – 618.
    (20) L. Kniznik, P.R. Alonso, P.H. Gargano , G.H. Rubiolo, Energetics and el
ectronic structure of UAl4 with point defects, Journal of Nuclear Materials 466 
(2015) 539-550.
    (21) M.D. Forti, P.R. Alonso, P.H. Gargano, P.B. Balbuena, G.H. Rubiolo, A D
FT study of atomic structure and adhesion at the Fe(BCC)/Fe3O4 interfaces, Surfa
ce Science 647 (2016) 55–65.
    (22)  P.H. Gargano, L. Kniznik, P.R. Alonso, M.D. Forti, G.H. Rubiolo, Conce
ntration of constitutional and thermal defects in UAl4, Journal of Nuclear Mater
ials 478 (2016) 74-82.
12.3  Premios recibidos 
    (1)  Mención especial a Mariano Forti en el CONCURSO ESTÍMULO A JÓVENES INVE
STIGADORES EN CIENCIA Y TECNOLOGÍA DE MATERIALES. Trabajo premiado: Forti M., Ba
lbuena P. y Alonso P., ESTUDIOS AB-INITIO SOBRE CARBURACIÓN EN ALEACIONES DE BAS
E Fe3Al, 11º Congreso Binacional de Metalurgia y Materiales SAM / CONAMET 2011, 
18 al 21 de Octubre de 2011 - Rosario, Argentina.
    (2)  PREMIO JORGE KITTL. Mejor trabajo en Investigación Básica en Ciencia de
 Materiales. Trabajo premiado: Paula R. Alonso, Pablo H.Gargano y Gerardo H. Rub
iolo, ESTADO FUNDAMENTAL DEL PSEUDO BINARIO Al3U-Si3U POR PRIMEROS PRINCIPIOS. S
OLUCIÓN SÓLIDA U(Al,Si)3, 11º Congreso Binacional de Metalurgia y Materiales SAM
 / CONAMET 2011, 18 al 21 de Octubre de 2011 - Rosario, Argentina.
    (3)  PREMIO JORGE KITTL. Mejor trabajo en Investigación Básica en Ciencia de
 Materiales. Trabajo premiado: Pedro A. Ferreirós, Paula R. Alonso, Pablo H. Gar
gano, Patricia B. Bozzano, Horacio E. Troiani y Gerardo H. Rubiolo, Transformaci
ones de fase en aleaciones Fe1-2XAlXVX (X0,15), XIII Congreso Internacional SAM
-CONAMET Iguazú 2013, 20 al 23 de agosto de 2013, Iguazú, Misiones, Argentina.
13. Proyectos científicos y/o tecnológicos de los que participaría esta tesis 
    (4) “Buscando una nueva aleación para el elemento combustible CAREM-25” – AN
PCyT, PICT-2015-2267. Titular: Gerardo H. Rubiolo. Grupo Responsable: Paula R. A
lonso, Pablo H. Gargano, Liliana A. Lanzani. Grupo Colaborador: Laura Kniznik, P
edro A. Ferreirós, Mariano D. Forti, Héctor A. Raffaelli, Sergio V. Ilarri, Mari
o A. Acosta, Perla B. Balbuena. Fecha de vigencia: 2016 – 2019.
    (5) “Buscando una nueva aleación para el elemento combustible CAREM-25” - Pr
oyecto UNSAM 80020160500046SM con vigencia Enero 2017-Diciembre 2019. Directora 
del proyecto: Paula Regina Alonso.
       Referencias













Mariano Forti
Laura Kniznik
